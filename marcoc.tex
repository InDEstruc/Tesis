%%%%%%%%%%%%%%%%%%%%%%%%%%%%%%%%%%%%%%%%%%%%%%%%%%%%%%%%%%%%%%%%%%%%%%%%%%%%%%%%%%%%%%%%%%%%%%%%%%%%%% MARCO CONTEXTUAL %%%%%%%%%%%%%%%%%%%%%%%%%%%%%%%%%

\chapter{Marco Contextual}

\begin{flushleft}
\begin{verse}
	\emph{\quotes{``Una cosa es continuar la historia y otra repetirla''}.}
\bigskip\newline
	 \qauthor{Jacinto Benavente.}
\end{verse}
\end{flushleft}

\section{Metodolog\'{\i}a Basada en el Desempe\~no}

El dise\~no basado en el \emph{desempe\~no s\'{\i}smico} es un moderno enfoque de dise\~no en el cual se busca que las estructuras sean no s\'olo capaces de resistir los eventos s\'{\i}smicos extremos esperados sin colapsar, sino que igualmente sean capaces de resistir fen\'omenos de menor intensidad s\'{\i}smica sin presentar da\~nos catastr\'oficos que impidan el uso de las estructuras.

La filosof\'{\i}a del desempe\~no como es recomendada por la Asociaci\'on de Ingenieros Estructurales de California (SEAOC, Vision $2000$) \cite{SEAOC1995}, y por la Gu\'{\i}a del Programa para la Reducci\'on de Riesgo S\'{\i}smico Nacional de los Estados Unidos (NEHRP) \cite{NEHRP1997} se basa en el equilibrio entre tres factores: la protecci\'on de vida, la reducci\'on de da\~no en las estructuras y la reducci\'on de p\'erdidas econ\'omicas. De esta manera, se establecen diversos niveles de desempe\~no seg\'un los tres factores considerados anteriormente; los niveles estar\'an determinados a su vez por la naturaleza de la zona en la que se encuentran las estructuras (peligro) y la finalidad de la estructura (importancia). 

\newpage

En primer lugar conviene destacar cu\'ales son dichos niveles de desempe\~no, para esto se refiere a la Tabla \ref{tab:desemp}; como puede observarse existen cuatro niveles b\'asicos en los que se espera un comportamiento referente al nivel del evento s\'{\i}smico. 

	\begin{table}[htbp]
	\begin{flushleft}	
		\caption{Descripcipci\'on de los Niveles de Desempe\~no Estructural Respecto al Nivel de Da\~no. Adaptado de \cite{ST2008}.} \newline
		\begin{tabular}{cccl}
			\hline \hline
	  \multicolumn{1}{c}{Estado} &	Gu\'{\i}a \ del  & Visi\'on 2000 & Descripci\'on  \\ 
	  \multicolumn{1}{c}{de Da\~no} & NEHRP & & \\
 \hline
	 \multicolumn{1}{c}{Despreciable} & Operacional  & Completamente& No se presenta da\~no significativo\\ 		  	
	\multicolumn{1}{c}{} & & Operacional & en componentes estructurales\\
				\multicolumn{1}{c}{} & & & o no. Disponible para\\
				\multicolumn{1}{c}{} & & & ocupaci\'on inmediata\\ \hline
	 \multicolumn{1}{c}{Ligero} & Ocupaci\'on & Operacional & No hay da\~no estructural,\\ 
				  	\multicolumn{1}{c}{} & Inmediata & & los elementos no estructurales\\
										\multicolumn{1}{c}{} & & &estan seguros. La estructura\\
															\multicolumn{1}{c}{} & & &esta en etapa funcional\\ \hline
	\multicolumn{1}{c}{Moderado}	&	Protecci\'on & Vida &Da\~no estructural significativo,\\
			  \multicolumn{1}{c}{} & de la Vida & Segura &los elementos no estructurales\\
								\multicolumn{1}{c}{} & & &pueden no funcionar.\\
								\multicolumn{1}{c}{} & & &Se requieren reparaciones.\\\hline
  	\multicolumn{1}{c}{Severo} & Prevenci\'on & Cercano &Da\~no estructural y no estructural \\
		  	\multicolumn{1}{c}{} & del Colapso & al Colapso & importante, reducci\'on\\
				\multicolumn{1}{c}{} & & &de resistencia y rigidez.\\
								\multicolumn{1}{c}{} & & &Peque\~no margen de colapso\\
			 	\multicolumn{1}{c}{} &  & &y gran cantidad de escombros.\\\hline \hline
		\end{tabular}
			\end{flushleft}
	\label{tab:desemp}
\end{table}

\newpage

En la Tabla \ref{tab:desempe} se establece la relaci\'on entre los niveles de desempe\~no buscados, como puede observarse existen a su vez tres clasificaciones principales de niveles buscados: para estructuras ordinarias, estructuras esenciales y estructuras de seguridad cr\'{\i}tica.

	\begin{table}[htbp]
	\centering
		\caption{Niveles de Desempe\~no Estructural Recomendados por la SEAOC. Adaptado de \cite{SEAOC1995}.}
		\begin{tabular}{||cc|c|c|c||}
			\hline \hline \multicolumn{1}{||l}{$0.-$Desempe\~no Inaceptable} & \multicolumn{4}{|c||}{}\\ 
			\multicolumn{1}{||l}{$1.-$Instalaciones \ B\'asicas} & \multicolumn{4}{|c||}{Niveles de Desempe\~no S\'{\i}smico}\\
			\multicolumn{1}{||l}{$2.-$Instalaciones \ Esenciales} & \multicolumn{4}{|c||}{}\\
			\multicolumn{1}{||l}{$3.-$Instalaciones \ Cr\'{\i}ticas} & \multicolumn{4}{|c||}{}\\ \hline 
			Nivel de Movimiento & \multicolumn{1}{|c|}{Totalmente} & Operacional & Vida& Cercano \\
		 S\'{\i}smico & \multicolumn{1}{|c|}{Operacional} & Operacional & Segura & al Colapso \\ \hline
			\multicolumn{1}{||c|}{Frecuente} & 1 & 0 & 0 & 0 \\ 
			\multicolumn{1}{||c|}{T=43 a\~nos} & & & & \\ \hline
			\multicolumn{1}{||c|}{Ocasional} & 2 & 1 & 0 & 0 \\						
			\multicolumn{1}{||c|}{T=72 a\~nos} & & & & \\ \hline
			\multicolumn{1}{||c|}{Raro} & 3 & 2 & 1 & 0 \\
						\multicolumn{1}{||c|}{T=475 a\~nos} & & & & \\ \hline
			\multicolumn{1}{||c|}{Muy Raro } & -- & 3 & 2 & 1 \\
			\multicolumn{1}{||c|}{T=970 a\~nos} & & & &  \\ \hline \hline
		\end{tabular}
	\label{tab:desempe}
\end{table}

La determinaci\'on de los \emph{niveles de desempe\~no} requiere en primer instancia de la determinaci\'on de dos propiedades caracter\'{\i}sticas de los sistemas estructurales: \emph{la capacidad s\'{\i}smica} y la \emph{demanda s\'{\i}smica}. De esta manera el desempe\~no consiste en la evaluaci\'on del nivel de superioridad de la capacidad sobre la demanda del sistema \cite{ATC1996}.

\newpage

En el caso de las estructuras con muros de mamposter\'{\i}a de relleno, los niveles de desempe\~no en la etapa de \emph{seguridad de vida} son primordiales debido a que la falla de los muros fuera del plano puede provocar derrumbes importantes afectando directamente la seguridad de los ocupantes. Por otro lado, los muros pueden provocar la formaci\'on de \emph{piso suave} participando directamente en la etapa de \emph{prevenci\'on de colapso}.

%%%%%%%%%%%%%%%%%%%%%%%%%%%%%%%%%%%%%%%%%%%%%%%%%%%%%%%%%%%%%%%%%%%%%%%%%%%%%%%%%%%%%%%%%%%%%%%%%%%%%%%%%%%%%%%%%%%%%%%%%%%%%%%%%%%%%%%%%%%%%%%%%%%%%%%%%%%%%%%%%%%%%%%%%%%%%%%%%%%%%%%%%%%%%%%%%%%%%%%%%%%%%%%%%%%%%%%%%%%%%%%%%%%%%%%%%%

	\section{Estado del Arte}

	\subsection{Investigaciones Experimentales}

A lo largo del siglo XX se realizaron un conjunto de investigaciones buscando estudiar de qu\'e manera realizar construcciones m\'as eficientes con el uso de la  mamposter\'{\i}a como elemento resistente a cargas laterales, lo anterior debido a que diversas investigaciones reportaban la contribuci\'on de los muros a la resistencia y rigidez de edificios.

Polyakov \cite{P1960} a trav\'es de trabajo experimental en muros de mamposter\'{\i}a a escala, presenta una propuesta de un modelo anal\'{\i}tico a trav\'es de puntales a lo largo de las diagonales del muro que unicamente soportan esfuerzos de compresi\'on. 

Fiorato, Sozen y Gamble \cite{FSG1970} estudian la influencia de los muros de relleno de mamposter\'{\i}a en los marcos de concreto reforzado, realizaron ocho modelos a escala de marcos rellenos de mamposter\'{\i}a sujetos a carga lateral y en algunos casos a carga vertical; definiendo los mecanismos de falla, la influencia de las variables de control, concluyendo que la resistencia y rigidez es mayor en los marcos rellenos de mamposter\'{\i}a a\'un con ciertas aberturas y diversas relaciones de aspecto en comparaci\'on con los marcos libres.

\newpage

Williams \cite{W1971} presenta en su tesis doctoral un estudio sobre el comportamiento s\'{\i}smico de muros de mamposter\'{\i}a de cortante reforzada considerando la degradaci\'on de rigidez y resistencia en muros de block ante diversas distribuciones de refuerzo aplicando cargas horizontales cuasi est\'aticas.

Tomazevic, Lutman y Petkovic \cite{TLP1996} realizan una investigaci\'on experimental extensa probando cuatro grupos de patrones de carga en $32$ muros de mamposter\'{\i}a de iguales caracter\'{\i}sticas a escala, se muestra la influencia de la variaci\'on en los patrones, velocidad de aplicaci\'on y combinaciones de cargas.

Crisafulli \cite{C1997} propone un modelo general del comportamiento hister\'etico de los muros de relleno de mamposter\'{i}a mediante una extensa cantidad de informaci\'on experimental proveniente principalmente de Am\'erica Latina, dicho modelo de comportamiento hister\'etico es incluido en el software Ruaumoko \cite{C2004}. 

Existen estudios recientes sobre muros divisorios de yeso con mortero de relleno expuestos ante cargas laterales \cite{YCC2012,TPP2012} donde se presenta informaci\'on de la contribuci\'on lateral de elementos de este tipo, dejando la posibilidad de que futuras investigaciones pueden estar orientadas en an\'alisis de confiabilidad y optimaci\'on considerando estos elementos no estructurales. 

\begin{figure}[htbp]
	\centering
		\includegraphics[scale=0.5]{figmc1.png}
	\caption{Muro a escala real de yeso sometido a cargas horizontales \cite{YCC2012}.}
	\label{fig:figmc1}
\end{figure}

\newpage

El comportamiento fuera del plano de los muros de mamposter\'{\i}a es elemento clave para la comprensi\'on global de la mamposter\'{\i}a; sin embargo, es un problema complejo de estudio reciente. Algunos investigadores sugieren que la estabilidad de los muros est\'a determinada por la capacidad de deformaci\'on pl\'astica por tensi\'on \cite{NBKD2012}.

\begin{figure}[htbp]
	\centering
		\includegraphics[scale=0.5]{figmc2.png}
	\caption{Muro a escala real sometido a cargas fuera del plano \cite{NBKD2012}.}
	\label{fig:figmc2}
\end{figure}

	\subsection{M\'exico y Am\'erica Latina}

Esteva L. \cite{E1961} fue uno de los primeros en realizar investigaciones sobre el comportamiento de los muros de mamposter\'{\i}a, en este trabajo el autor remarca la importante necesidad de realizar investigaciones experimentales con materiales y procesos constructivos del pa\'{\i}s para obtener resultados representativos.

Meli \cite{M1979} realiz\'o un estudio de las propiedades mec\'anicas del mortero y la mamposter\'{\i}a estudiando los modos de falla de mamposter\'{\i}a bajo compresi\'on y cortante en ensambles peque\~nos de mamposter\'{\i}a teniendo as\'{\i} informaci\'on estad\'{\i}stica, a su vez estudio muros a escala ante cargas laterales en una direcci\'on y c\'{\i}clicas alternadas generando un modelo de comportamiento hister\'etico con los datos obtenidos.

Meli \cite{M1994} analiza las caracter\'{\i}sticas estructurales de la mamposter\'{\i}a en M\'exico y los diversos beneficios de su uso ante cargas s\'{\i}smicas.

Alcocer \emph{et al} \cite{ASVD1994} realizan una investigaci\'on experimental para determinar los patrones de falla y el comportamiento general de la mamposter\'{\i}a ante cargas s\'{\i}smicas con diversos grados de acoplamiento. 

\begin{figure}[htbp]
	\centering
		\includegraphics[scale=0.7]{figmc4.png}
	\caption{Esquema de carga horizontal en el plano \cite{AA2001}.}
	\label{fig:figmc3}
\end{figure}

Aguilar y Alcocer \cite{AA2001} determinan las caracter\'{\i}sticas de dise\~no y construcci\'on de muros de mamposter\'{\i}a con diversos tipos de confinamiento y con el empleo del refuerzo horizontal, se presenta el comportamiento de los muros reforzados de esta manera para resistir cargas s\'{\i}smicas; los resultados permiten desarrollar diversas normativas NTC-$2004$ \cite{NTC2004}. El comportamiento de las probetas de estudio M$2$ y M$4$ constituye la principal fuente de informaci\'on para este proyecto.

\subsection{Modelaci\'on de la Mamposter\'{\i}a.}

Debido a la naturaleza propia de la mamposter\'{\i}a, las investigaciones en M\'exico y el mundo se han realizado de manera dispersa y han servido para satisfacer diversas necesidades donde se recurre a una variedad de metodolog\'{\i}as para la consideraci\'on de la aplicaci\'on de cargas, de las dimensiones y de otras variables de inter\'es. Como consecuencia existen gran cantidad de modelos num\'ericos y anal\'{\i}ticos de la mamposter\'{\i}a. Por lo anterior, la selecci\'on de un modelo que simule las propiedades mec\'anicas de los muros es un factor determinante en la modelaci\'on de cualquier estructura con alg\'un tipo de mamposter\'{\i}a.

\begin{figure}[htbp]
	\centering
		\includegraphics[scale=0.93]{figmc5.png}
	\caption{Propiedades de dise\~no del esp\'ecimen M$2$ \cite{AA2001}.}
	\label{fig:figmc4}
\end{figure}

Uno de los proyectos en el cual se realiza tanto experimentaci\'on y modelaci\'on num\'erica es el de Hashemi y Mosalam \cite{HM2007}; se obtienen resultados de un modelo estructural de prueba elaborado de concreto reforzado con muros de mamposter\'{\i}a no reforzada de relleno a escala, el cual es sometido a cargas din\'amicas por mesa vibradora. Con los resultados experimentales y de modelaci\'on por \emph{elementos finitos no lineales} se calibra y presenta un modelo tipo \textit{puntal--tensor} tridimensional que considera la representaci\'on de efectos dentro y fuera del plano.

\begin{figure}[htbp]
	\begin{flushleft}
		\includegraphics[scale=0.57]{figmc6.pdf}
	\caption{Programa de estudio para marcos de concreto reforzado con muros no reforzados de mamposter\'{\i}a de relleno. Adaptado de Hashemi $&$ Mosalam \cite{HM2007}.}
	\label{fig:figmc5}
		\end{flushleft}
\end{figure}

\newpage

Posteriormente como parte del proyecto realizado por Hashemi y Mosalam; Kadysiewski y Mosalam \cite{SM2009} presentan un modelo modificado para representar los muros de mamposter\'{\i}a no reforzada de relleno idealizados mediante un elemento de una sola diagonal; en el modelo se utilizan elementos viga--columna de secci\'on discretizada por fibras con la teor\'{\i}a de desplazamientos peque\~nos. Este es el modelo utilizado para realizar el proceso de confiabilidad s\'{\i}smica en esta investigaci\'on.

P\'erez Mart\'{\i}nez \cite{Put2010} realiza un modelo de comportamiento hister\'etico de muros de mamposter\'{\i}a con y sin refuerzo horizontal conforme  los resultados experimentales de Aguilar y Alcocer \cite{AA2001}, considera la degradaci\'on de rigidez y resistencia de los elementos basados en el da\~no acumulado propuesto por Wang \cite{WS1987}. Debido a problemas de la idealizaci\'on tridimensional de la mamposter\'{\i}a este modelo hister\'etico no puede utilizarse para fines de la presente investigaci\'on.
	
	\subsection{El Piso Suave}
	
Chopra et al \cite{CCC1973} investigan el problema de piso suave considerando la fuerza de fluencia del primer nivel y por otro lado la rigidez de los entrepisos sometiendo un conjunto de estructuras seleccionadas a $20$ sismos artificiales con caracter\'{\i}sticas espec\'{\i}ficas; el objetivo de considerar estas variables fue correlacionar la reducci\'on de resistencia del primer nivel sobre las deformaciones y fuerzas desarrolladas en los pisos superiores demostrando que los pisos superiores permanecen dentro del rango de comportamiento el\'astico mientras que la planta baja sea capaz de desarrollar una capacidad de deformaci\'on cercana a los $30$ cm, esto es una demanda d\'uctil bastante grande para seguir en el rango el\'astico.

Ruiz y Diederich \cite{RD1988} como consecuencia de los efectos del sismo del $19$ de septiembre en M\'exico estudian la influencia sobre la respuesta din\'amica de las estructuras debido a los muros de relleno ante acelerogramas de banda angosta, espec\'{\i}ficamente de la relaci\'on de amplitud m\'axima y el periodo dominante de vibraci\'on mostrando que el comportamiento del piso suave en planta baja se encuentra influenciado por: la relaci\'on entre el periodo dominante y el de la respuesta, la relaci\'on existente de resistencia entre el piso inferior y los superiores y por el coeficiente s\'{\i}smico de dise\~no debido a que la resistencia estructural global cambia. Recomiendan realizar an\'alisis paso a paso y correctos modelos de respuesta de los elementos de la estructura.

Esteva \cite{E1992} realiza despu\'es de los desastres mostrados en edificios de piso suave en el sismo de $1985$ una investigaci\'on del problema considerando como variables el n\'umero de niveles, el periodo fundamental de vibraci\'on, la variaci\'on de rigidez a lo alto del edificio y la relaci\'on de la rigidez posfluencia con la inicial; sin embargo, como variable principal incluye el factor $r$ que expresa el cociente del valor medio del factor de seguridad ante cortante lateral en los pisos superiores entre el del primer piso, como principal observaci\'on de los an\'alisis no lineales la respuesta s\'{\i}smica de edificios con piso suave es muy sensible a la raz\'on $r$ la cual se acent\'ua al considerar efectos P-delta, por otro lado la respuesta es igualmente sensible a la relaci\'on entre las rigideces posfluencia e inicial y a la correspondencia entre los periodos de vibrar. Recomienda finalmente realizar estudios que incluyan incertidumbres en las propiedades, estudios a base de sistemas de marcos con muros de relleno, entre otros.

Kappos y Ellul \cite{KE2000} realizan una investigaci\'on utilizando el Euroc\'odigo $8$ para estructuras de concreto reforzado con diferentes arreglos de distribuci\'on de mamposter\'{\i}a; uno de ellos corresponde al piso suave, como resultado de la aplicaci\'on de una metodolog\'{\i}a propuesta; observa que la mamposter\'{\i}a en estos casos es el mayor disipador de energ\'{\i}a seguido de los elementos viga, adem\'as se observan las debilidades del Euroc\'odigo $8$ para estructuras con piso suave en planta baja.

Galli \cite{G2006} estudia estructuras de concreto reforzado con y sin muros de relleno de mamposter\'{\i}a construidas previamente a los a\~nos $70$ en los pa\'{\i}ses del mediterr\'aneo a trav\'es de an\'alisis no lineales elaborados en el programa Ruaumoko \cite{C2004} resaltando la importancia de una correcta distribuci\'on de los muros y de un modelo refinado de uni\'on viga-columna; adem\'as plantea la necesidad de realizar trabajos futuros en tres dimensiones de este tipo de estructuras para establecer modelos m\'as representativos.

Madan y Hashmi \cite{MH2008} realizan una evaluaci\'on anal\'{\i}tica de los efectos de la distribuci\'on irregular en elevaci\'on de los muros de mamposter\'{\i}a en marcos de concreto reforzado a trav\'es de la metodolog\'{\i}a de dise\~no s\'{\i}smico por desempe\~no con sismos de origen cercano utilizando marcos planos, resaltan la necesidad de establecer criterios reglamentarios para sismos distintos en los sistemas con piso suave.

D�az Alc\'antara \cite{D2008} presenta una metodolog\'{\i}a general para llevar a cabo el estudio por desempe\~no y confiabilidad de sistemas de concreto reforzado con mamposter\'{\i}a en piso suave a trav\'es de la cual demuestra la perdida de niveles de confiabilidad debido a la formaci\'on de piso suave.

Existen  estudios experimentales recientes de estructuras de concreto reforzado con piso suave en planta baja; entre ellos el estudio de Guo, Zheng \textit{et al}, donde investigan los mecanismos de falla de varios modelos con piso suave en planta baja desarrollados a escala $1/5$ ante cargas s\'{\i}smicas simuladas por mesa vibradora; se identifican los beneficios de colocar elementos de alta rigidez como compensaci\'on al problema de piso suave en planta baja \cite{XZH2012}.

\begin{figure}
	\centering
		\includegraphics[scale=0.45]{figmc3.png}
	\caption{Modelos a escala estudiando el efecto del piso suave con y sin elementos r\'{\i}gidos de compensaci\'on \cite{XZH2012}.}
	\label{fig:figmcmm2}
\end{figure}

\newpage

Surya Kumar Dadi $&$ Pankaj \cite{SP2012} realizan modelos a escala $1/4$ de un sistema de marcos de concreto reforzado con muros de mamposter\'{\i}a con piso suave en planta baja, en los modelos se busca variar el tipo de refuerzo, se utilizan dos tipos: refuerzo con tratamiento mec\'anico t\'ermico y refuerzo de alto l\'{\i}mite el\'astico, se realiza el detallado conforme diversas normas indias de concreto reforzado (IS $456:2000$ y IS $1893:2002$) lo anterior con la finalidad de observar la ductilidad del sistema resultante someti\'endolo a cargas c\'{\i}clicas; se establecen las ventajas sobre la capacidad de deformaci\'on en el rango no lineal respecto el tipo de refuerzo y norma aplicada.

P\'erez Mart\'{\i}nez y Esteva \cite{PE2012} realizaron la aplicaci\'on de la metodolog\'{\i}a seguida por Esteva \emph{et al} \cite{EDGSI2002}  para tres sistemas estructurales incluyendo ciertas distribuciones de muros de mamposter\'{\i}a con y sin refuerzo horizontal en marcos de concreto reforzado considerando el modelo hister\'etico de P\'erez Mart\'{\i}nez \cite{Put2010} para mamposter\'{\i}a con y sin refuerzo horizontal; uno de los casos explora las caracter\'{\i}sticas del piso suave en planta baja; el estudio se realiza a trav\'es de un an\'alisis de marcos planos.   

Esteva \emph{et al} \cite{CTEM2013} realizan una propuesta para la obtenci\'on de funciones de confiabilidad en estructuras considerando un marco de referencia tridimensional; como parte de esta investigaci\'on se continua el planteamiento de sistemas simplificados de referencia (SSR) que permiten disminuir la demanda computacional pero cabe destacar que en esta investigaci\'on no se obtienen resultados definitivos para el uso de un SSR adecuado. Se sugiere el empleo de modelos m\'as precisos del comportamiento global del sistema detallado.

\begin{figure}[htbp]
	\centering
		\includegraphics[scale=0.8]{fig15.png}
	\caption{Modelo a escala estudiando el efecto del piso suave \cite{SP2012}.}
	\label{fig:fig15}
\end{figure}

Algunos investigadores sugieren que la capacidad rotacional de las columnas influye directamente en la capacidad final de desplazamientos globales en las estructuras con piso suave en planta baja \cite{PMGG2012}.
	
	

%%%%%%%%%%%%%%%%%%%%%%%%%%%%%%%%%%%%%%%%%%%%%%%%%%%%%%%%%%%%%%%%%%%%%%%%%%%%%%%%%%%%%%%%%%%%%%%%%%%%%%%%%%%%%%%%%%%%%%%%%%%%%%%%%%%%