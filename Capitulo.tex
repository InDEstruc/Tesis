                                                                        %%%%%%%%%% RESULTADOS DE LA INVESTIGACI\'ON %%%%%%%%%%

\chapter{ An\'alisis de Resultados}

\begin{flushleft}
\begin{verse}
	\emph{\quotes{``Para las personas creyentes, Dios esta al principio. Para los cient\'{\i}ficos est\'a el final de todas sus reflexiones''}.}
\newline
	 \qauthor{Max Planck.}
\end{verse}
\end{flushleft}

\section{Resultados de la Simulaci\'on por MonteCarlo}

Como parte de los resultados de las simulaciones de propiedades es posible observar en la Figura \ref{fig:dim2} y \ref{fig:rf1} las diferencias existentes entre los periodos de vibrar en uno u otro sentido cuyo origen reside en los cambios existentes entre las secciones simuladas.

\begin{figure} [htbp]
\includegraphics[width=150mm]{COMPCMXZM1.pdf}
\caption{Comparativa de los resultados de los modos de vibrar.}
\label{fig:dim2}
\end{figure}

\begin{figure} [htbp]
\includegraphics[width=150mm]{COMPSMXZM1.pdf}
\caption{Comparativa de los resultados de los modos de vibrar.}
\label{fig:rf1}
\end{figure}

Como parte de esta investigaci\'on se realizaron ocho simulaciones independientes de edificios considerando cuatro simulaciones con el modelo de carga viva de Mitchel, G. R. y Woodgate, R. W \cite{MG1970} y otras cuatro con el de Peir J. y Cornell C. \cite{PC1973}. Las conclusiones de esta parte de la investigaciones pueden obtenerse de manera detallada en uno de los art\'{\i}culos obtenidos por esta investigaci\'on \cite{SIBCV2014}.

\begin{figure} [htbp]
\includegraphics[width=150mm]{GCV1.pdf}
\caption{Distribuci\'on de carga viva en entrepiso conforme al modelo de Mitchelll y Goodwate, en $kg/m^{2}$.}
\label{fig:rf2}
\end{figure}

\begin{figure} [htbp]
\includegraphics[width=150mm]{GCV2.pdf}
\caption{Distribuci\'on de carga viva en entrepiso conforme al modelo de Pier y Cornell, en $kg/m^{2}$.}
\label{fig:rf3}
\end{figure}

\newpage

En la Figura \ref{fig:rf2} y \ref{fig:rf3} se presenta una comparativa resultante de la distribuci\'on de cargas vivas resultado de los dos modelos de simulaci\'on estudiados. Como parte de los resultados de estos tipos de modelos es posible observar la ventaja del modelo de Peir J. y Cornell C. \cite{PC1973} respecto al de Mitchel, G. R. y Woodgate, R. W \cite{MG1970} debido a la correlaci\'on implicita entre \'areas, edificios y niveles; lo anterior resulta en un modelo con gradientes suavizados entre diversos puntos del entrepiso.

\section{Modelado No Lineal de la Mamposter\'{\i}a.}

Una parte fundamental del modelado no lineal de la mamposter\'{\i}a como se ha venido presentando es la consideraci\'on de los efectos fuera del plano del modelo de la mamposter\'{\i}a de manera que resulta de vital importancia observar que el comportamiento sea el adecuado de acuerdo a las simplificaciones del an\'alisis; para mayor informaci\'on vease el \emph{Ap\'endice \textbf{A}}. En la Figura \ref{fig:rf4} a \ref{fig:rf8} se presenta visualmente algunos resultados de la interacci\'on de los desplazamientos, es posible concluir que siempre se mantienen dentro de los l\'{\i}mites del diagrama calculado. En la Figura \ref{fig:rf8} puede observarse la tendencia de cambio de los desplazamientos conforme se incrementa en entrepiso, lo anterior es resultado de los modos de vibrar de la estructura.

\begin{figure} [htbp]
\includegraphics[width=150mm]{ID2SM6P1.pdf}
\includegraphics[width=150mm]{ID2SM6P5.pdf}
\caption{Comportamiento dentro y fuera del plano del modelo de mamposter\'{\i}a ante Pushover.}
\label{fig:rf4}
\end{figure}

\begin{figure} [htbp]
\includegraphics[width=150mm]{IDM5S2N2.pdf}
\includegraphics[width=150mm]{IDM5S2N3.pdf}
\caption{Comportamiento dentro y fuera del plano del modelo de mamposter\'{\i}a ante Pushover.}
\label{fig:rf5}
\end{figure}

\begin{figure} [htbp]
\includegraphics[width=150mm]{IDM1S1.pdf}
\label{fig:rf7}
\end{figure}

\begin{figure} [htbp]
\includegraphics[width=150mm]{IDM1S2.pdf}
\caption{Comportamiento dentro y fuera del plano del modelo de mamposter\'{\i}a ante Sismos.}
\label{fig:rf8}
\end{figure}

Otra de las ventajas del modelo aplicado de la mamposter\'{\i}a es la remoci\'on del elemento debida al da\~no excesivo. En la Figura \ref{fig:apendis24} y \ref{fig:apendis25} del \emph{Apendice \textbf{C}} es posible observar el orden determinado de ubicaci\'on de los muros de mamposter\'{\i}a. Por otro lado un ejemplo de la secuencia de eliminaci\'on de los muros sujetos  a los diversos an\'alisis se muestra en la Figura \ref{fig:rf11}; en dicha Figura es posible observar la eliminaci\'on de los muros correspondientes al segundo nivel y el incremento de desplazamientos relativos conforme se incrementa el n\'umero de entrepiso, el muro uno del segundo entrepiso corresponde al muro $13$ del tercero, al $25$ del cuarto y al $37$ del quinto entrepiso.

\begin{figure} [htbp]
\includegraphics[width=150mm]{SEP1SM6.pdf}
\caption{Proceso de remoci\'on de mamposter\'{\i}a ante Pushover.}
\label{fig:rf11}
\end{figure}

\newpage

En el \emph{Apendice \textbf{D}} se muestran los casos complementarios de eliminaci\'on de muros; con base en estos resultados es posible concluir que conforme al orden de numeraci\'on de los muros de mamposter\'{\i}a la eliminaci\'on resulta acorde a lo esperado puesto que en la aplicaci\'on del Pushover los muros que reciben mayor demanda y consecuentemente se eliminan son los que se encuentran paralelos a la direcci\'on de aplicaci\'on de la carga, mientras que en los an\'alisis din\'amicos los muros que se eliminan dependen de las formas de vibrar de la estructura. Respecto a las gr\'aficas de eliminaci\'on cabe mencionar que en el caso de los empujes laterales, los muros del primer nivel son eliminados y conforme la estructura sufre mayores da\~nos se eliminan los muros superiores, sin embargo, estos corresponden a las cargas descendentes del comportamiento no lineal y no a las correspondientes a la eliminaci\'on de los muros del segundo nivel.

\newpage

\section{Modelado No Lineal de la Estructura.}

A continuaci\'on se presenta una parte de los resultados de la modelaci\'on no lineal de la estructura; existe una gran cantidad de informaci\'on disponible para caracterizar el comportamiento global de la estructura, sin embargo, unicamente se utilizan en esta investigaci\'on los datos m\'as significativos que permitan representar el comportamiento global de los edificios.

\begin{figure} [htbp]
\includegraphics[width=150mm]{GX1E6CAR1.pdf}
\caption{Gr\'afica de un an\'alisis din\'amico no lineal de un sismo con sus tres componentes de aceleraci\'on escaladas un factor de tres.}
\label{fig:rfadnl1a}
\end{figure}

\begin{figure} [htbp]
\includegraphics[width=150mm]{GX1E6CAR1CM.pdf}
\caption{Gr\'afica de un an\'alisis din\'amico no lineal de un sismo con sus tres componentes de aceleraci\'on escaladas un factor de tres, continuaci\'on.}
\label{fig:rfadnl1b}
\end{figure}

\begin{figure} [htbp]
\includegraphics[width=150mm]{GX1E5CAR1.pdf}
\caption{Gr\'afica de un an\'alisis din\'amico no lineal de un sismo con sus tres componentes de aceleraci\'on escaladas un factor de dos y medio.}
\label{fig:rfadnl2a}
\end{figure}

\begin{figure} [htbp]
\includegraphics[width=150mm]{GX1E5CAR1CM.pdf}
\caption{Gr\'afica de un an\'alisis din\'amico no lineal de un sismo con sus tres componentes de aceleraci\'on escaladas un factor de dos y medio, continuaci\'on.}
\label{fig:rfadnl2b}
\end{figure}

\begin{figure} [htbp]
\includegraphics[width=150mm]{GX1E4CAR1.pdf}
\caption{Gr\'afica de un an\'alisis din\'amico no lineal de un sismo con sus tres componentes de aceleraci\'on escaladas un factor de dos.}
\label{fig:rfadnl3a}
\end{figure}

\begin{figure} [htbp]
\includegraphics[width=150mm]{GX1E4CAR1CM.pdf}
\caption{Gr\'afica de un an\'alisis din\'amico no lineal de un sismo con sus tres componentes de aceleraci\'on escaladas un factor de dos, continuaci\'on.}
\label{fig:rfadnl3b}
\end{figure}

\newpage

El comportamiento din\'amico de la estructura converge en un principio sobre un esquema el\'astico, el cual posteriormente al sufrir mayor da\~no se desplaza a valores de rigidez menores a los iniciales. Las diferencias primordiales entre las estructuras tipo ECMD y ECML en los an\'alisis din\'amicos residen en que los muros transmiten mayor rigidez globalmente a la estructura pero la rigidez de las columnas en planta baja retrasa la formaci\'on de un piso suave hasta altos grados de da\~no estructural.

\begin{figure} [htbp]
\includegraphics[width=150mm]{GFR1Z.pdf}
\includegraphics[width=150mm]{GFR1CMZ.pdf}
\caption{Comportamiento de los nodos maestros de entrepiso en Pushover de propiedades medias en la direcci\'on $+Z$.}
\label{fig:rfadnl4}
\end{figure}

\begin{figure} [htbp]
\includegraphics[width=150mm]{GFR2X.pdf}
\includegraphics[width=150mm]{GFR2CMX.pdf}
\caption{Comportamiento de los nodos maestros de entrepiso en Pushover de propiedades medias en la direcci\'on $+Z$, continuaci\'on.}
\label{fig:rfadnl5}
\end{figure}

\newpage

Como puede observarse en la Figura \ref{fig:rfadnl4} y \ref{fig:rfadnl5} existen diferencias significativas unicamente en los entrepisos menores al superior, principalmente la tendencia de falla se posiciona en el entrepiso n\'umero uno y dos, posiblemente por la presencia de los muros de mamposter\'{\i}a y el cambio de rigidez.

El tipo de pushover aplicado posiblemente influye en la forma de falla debido a que la aplicaci\'on de fuerzas es controlada por los desplazamientos del nodo maestro del nivel superior de manera que  las curvas de cortante basal contra desplazamientos utilizadas para la obtenci\'on del IRRS resultan de gran similitud entre si. Cabe mencionar que los valores del cortante basal utilizados corresponden a los valores de las fuerzas resultantes del plano, es decir, de los ejes X y Z as\'{\i} como los desplazamientos considerados son los desplazamientos resultantes.

\begin{figure} [htbp]
\includegraphics[width=150mm]{GFR3R.pdf}
\caption{Rotaci\'on de los nodos maestros de entrepiso en Pushover de propiedades medias en la direcci\'on $+Z$.}
\label{fig:rfadnl6}
\end{figure}

\begin{figure} [htbp]
\includegraphics[width=150mm]{GFR4CM.pdf}
\includegraphics[width=150mm]{GFR4SM.pdf}
\caption{Distorsi\'on de los nodos maestros de entrepiso en Pushover de propiedades simuladas en la direcci\'on $+Z$.}
\label{fig:rfadnl7}
\end{figure}

\newpage

En los resultados puede observarse que debido a la generaci\'on de los mecanismos de falla en la direcci\'on de aplicaci\'on de la carga, la estructura se re-acomoda liberando energ\'{\i}a en estos puntos y exigiendo demanda sobre la direcci\'on perpendicular, por lo anterior como medida de da\~no directo podr\'{\i}a resultar adecuado el uso de las rotaciones y distorsiones de entrepiso las cuales son visualmente distintas entre si. T\'omese como ejemplo la Figura \ref{fig:rfadnl6} y \ref{fig:rfadnl7} en las cuales se puede observar que existe valores variables de simulaci\'on en simulaci\'on. 

\begin{figure} [htbp]
\includegraphics[width=150mm]{GFR5CM.pdf}
\includegraphics[width=150mm]{GFR5CMZOOM.pdf}
\includegraphics[width=150mm]{GFR5SM.pdf}
\caption{Diferencias en las curvas de Pushover en la direcci\'on $+Z$ en sistemas ECMD y ECML.}
\label{fig:rfadnl8}
\end{figure}

Finalmente resulta importante observar que las pendientes iniciales mostradas en la Figura \ref{fig:rfadnl8} son b\'asicamente equivalentes de manera que la pendiente de capacidad y da\~no requeridas por el $IRRS$ no presentan distinci\'on considerable.

%%%%%%%%%%%%%%%%%%%%%%%%%%%%%%%%%%%%%%%%%%%%%%%%%%%%%%%%%%%%%%%%%%%%%%%%%%%%%%%%%%%%%%%%%%%%%%%%%%%%%%%%%%%%%%%%%%%%%%%%%%%%%%%%%%%%%%%%%%%%%%%%%%%%%%%%%%%%