%%%%%%%%%%%%% LA MAMPOSTER�A %%%%%%%%%%%%%%%%%

\chapter{Construcciones de Mamposter\'{\i}a}

\begin{flushleft}
\begin{verse}
	\emph{\quotes{``El experimentador que no sabe lo que est\'a buscando no comprender\'a lo que encuentra''}.}
\newline
	 \qauthor{Claude Bernard.}
\end{verse}
\end{flushleft}

\section{La Mamposter\'{\i}a en el Tiempo}

Desde los primeros pasos de las civilizaciones humanas, la mamposter\'{\i}a fue elemento clave en el desarrollo de un lugar adecuado para sobrellevar las inclemencias naturales, incluso hoy en d\'{\i}a algunas de estas construcciones contin\'uan asombrando por su excelente conservaci\'on, para esto basta con nombrar la gran muralla China ($1500$ a. c.), las pir\'amides de Egipto ($2500$ a. c.), de M\'exico y Centroam\'erica ($500$ a. c.). Debido a que a pesar de las inclemencias naturales estas construcciones antiguas siguen en pie, el estudio de su comportamiento es elemento clave en la revisi\'on de las propuestas estructurales de la actualidad, A. Bayraktar \emph{et al} \cite{ANC2012} realizan un estudio sobre diversas de estas construcciones de relativa antig\"uedad con la finalidad de entender su funcionamiento y aprovecharlo.

A continuaci\'on se presentan en orden cronol\'ogico ciertas caracter\'{\i}sticas del desarrollo de la mamposter\'{\i}a en diversas culturas, esperando ampliar la percepci\'on de este sistema constructivo; para profundizar en el tema se recomienda revisar las referencias \cite{ICA12003} y \cite{HE2008}  mismos que se sirven de fuente en este cap\'{\i}tulo.

%%%%%%%%%%%%%%%%%%%%%%%%%%%%%%%%%%%%%%%%%%%%%%%%%%%%%%%%%%%%%%%%%%%%%%%%%%%

\subsection{Amanece en la Prehistoria}

Probablemente uno de los primeros elementos que utiliz\'o el hombre primitivo buscando moldear un lugar adecuado para habitar fueron las rocas que encontraba disponibles en su entorno, posteriormente recurri\'o al uso adicional de mezclas de ciertos suelos (morteros de barro) para mantener uniones entre las rocas con grandes irregularidades; dicha mezcla servir\'{\i}a despu\'es para moldear a mano elementos semejantes entre s\'{\i}. La elaboraci\'on de piezas regulares hizo posible un acomodo m\'as estable y esto logr\'o reducir as\'{\i} la dificultad de obtener las rocas.

\begin{figure}
	\centering
		\includegraphics{fig1.png}
	\caption{Yacimiento arqueol\'ogico  Dun Dubhchathair, islas de Aran (Irlanda). \newline }
	\label{fig:fig1}
\end{figure}

Existen vestigios del uso de rocas como mamposter\'{\i}a en poblados que van desde las islas Aran, en Irlanda, hasta Catal H\"uy\"uk en Anatolia, cerca de $10 000$ a\~nos despu\'es el mismo sistema constructivo fue empleado por los Incas en Ollantaytambo cerca de Cusco en lo que corresponde actualmente a Per\'u. 

\newpage

Las unidades de barro formadas a mano m\'as antiguas se han encontrado en Jeric\'o en el medio oriente y en otras partes en formas muy diversas, siendo la c\'onica una de las formas m\'as comunes de las cuales se han encontrado en conjuntos de muros en Mesopotamia con una antig\"uedad cercana a los $7 000$ a\~nos y en las costas de Per\'u, en huaca prieta con una antig\"uedad de $5 000$ a�os. 

%%%%%%%%%%%%%%%%%%%%%%%%%%%%%%%%%%%%%%%%%%%%%%%%%%%%%%%%%%%%%%%%%%%%%%%%%%%

\subsection{Primeros Pasos en Sumeria}

Los sumerios (siglo IV a. C.) son considerados en la actualidad los iniciadores de la civilizaci\'on y de los primeros desarrollos ingenieriles, a tal grado que se consideran los inventores de la ciudad; t\'{\i}tulo que  se ganaron desarrollando la arquitectura del adobe a escala monumental y las construcciones tipo arco a trav\'es del uso de moldes de madera en forma de paralelep\'{\i}pedo para la fabricaci\'on de las piezas de mamposter\'{\i}a de adobe con paja secadas al sol.

\begin{figure}
	\centering
		\includegraphics[scale=0.31]{fig2.png}
	\caption{El Gran Zigurat Neo--Sumerio de Ur (Irak).}
	\label{fig:fig2}
\end{figure}

\newpage

Se reconoce tambi\'en a los sumerios la invenci\'on de los ladrillos cer\'amicos al llevar el adobe al horno (siglo III a. C.). Para la construcci\'on se utilizaba mortero de bet\'un o alquitran mezclado con arena, sustancias abundantes en la regi\'on; en construcciones elevadas el mortero era reforzado con fibras de ca\~na aumentando la resistencia a la tensi\'on. 

Como muestra de la importancia de las construcciones en Babilonia, los ladrillos cer\'amicos ten\'{\i}an inscripciones en bajo relieve que relataban tanto aspectos importantes de la obra como historias de \'esta y de los autores; dichos ladrillos terminada la obra eran esmaltados.

%%%%%%%%%%%%%%%%%%%%%%%%%%%%%%%%%%%%%%%%%%%%%%%%%%%%%%%%%%%%%%%%%%%%%%%%%%%

\subsection{Culturas de Mesoam\'erica}

Las construcciones de las civilizaciones mesoamericanas cuentan con rasgos caracter\'{\i}sticos generales, sin embargo, la gran variedad de culturas y creencias religiosas se vieron reflejadas en una gran variedad de estilos arquitect\'onicos y t\'ecnicas constructivas. 

Entre las civilizaciones mesoamericanas resalta la civilizaci\'on maya, la que entre numerosos avances constructivos desarrollo toda una tecnolog\'{\i}a de mamposter\'{\i}a de tabique de arcilla cocida similar a la usada en la actualidad en M\'exico como puede observarse en la zona arqueol\'ogica de Comalco en Tabasco.

\begin{figure}
	\centering
		\includegraphics[scale=0.4]{fig3.png}
	\caption{Zona arqueol\'ogica de Comalcalco en Tabasco.}
	\label{fig:fig3}
\end{figure}

Las culturas mesoamericanas descubrieron la actividad puzol\'anica de diferentes materiales como el nejayote residuo de la nixtamalizaci\'on del ma\'{\i}z, las cenizas volc\'anicas, las arcillas calcinadas y molidas finamente, se utilizaron a su vez diversos agregados naturales tales como la piedra p\'omez y las piedras volv\'anicas. 

\newpage

En el Taj\'{\i}n, existen vestigios de edificios cubiertos con grandes losas de concreto ligero sin refuerzo resultado de una correcta proporci\'on y una buena fabricaci\'on del concreto. Para la uni\'on de elementos en la mayor\'{\i}a de los casos fue utilizado un mortero de cal y arena, aunque existe evidencia del uso de mezclas de resina vegetal y arena.

%%%%%%%%%%%%%%%%%%%%%%%%%%%%%%%%%%%%%%%%%%%%%%%%%%%%%%%%%%%%%%%%%%%%%%%%%%%

\subsection{Manifestaciones en Egipto y Grecia}


En Egipto y Grecia el sistema constructivo de preferencia fue la mamposter\'{\i}a cicl\'opea. En Egipto las rocas utilizadas eran llevadas a su lugar de trabajo por medio de balsas cruzando el rio Nilo desde las monta\~nas; la mamposter\'{\i}a usada era de piedras calizas, areniscas, granitos, basaltos y albastros que eran asentadas con morteros de yeso y cal.

\begin{figure}[htbp]
	\centering
		\includegraphics[scale=0.4]{fig4.png}
	\caption{Piramides de Guiza.}
	\label{fig:fig4}
\end{figure}

\newpage

Grecia  bas\'o su arquitectura en elementos de mamposter\'{\i}a de piedra caliza asentados con morteros de cal y recubiertos de m\'armol, el cual se encontraba disponible en la regi\'on.

\begin{figure}[htbp]
	\centering
		\includegraphics[scale=0.4]{fig5.png}
	\caption{El teatro de Herodes �tico (Atenas).}
	\label{fig:fig5}
\end{figure}

%%%%%%%%%%%%%%%%%%%%%%%%%%%%%%%%%%%%%%%%%%%%%%%%%%%%%%%%%%%%%%%%%%%%%%%%%%%

\subsection{La Experiencia Romana}

Los romanos crearon una arquitectura diversa, en sus construcciones aprovecharon desde las mejores canteras egipcias, hasta piedras de sus dep\'ositos de caliza, travertino y tufa volc\'anica, en ocasiones utilizaron el lujoso m\'armol griego e incluso la mamposter\'{\i}a de arcilla de los sumerios.

La civilizaci\'on romana desarrollo diferentes sistemas constructivos como fruto del descubrimiento de un compuesto  equivalente al concreto puzol\'anico moderno, sus uso permiti\'o acelerar los procesos de endurecimiento de las construcciones facilitando el crecimiento de la infraestructura del imperio. Esta sustancia era conocida por los romanos como arena de Putuoli puesto que fue encontrada en la vecindad de Baia y Putuoli y en los alrededores del monte Vesubio seg\'un se narra en el conocido relato de Vitruvio ($25$ a.C.). 

El uso de este cemento facilit\'o la construcci\'on del arco, la b\'oveda y la c\'upula; la construcci\'on de cimentaciones competentes y la mejora de las bases de pavimentos fueron otros avances constructivos producto su uso. 

\begin{figure}[htbp]
	\centering
		\includegraphics[scale=0.4]{fig6.png}
	\caption{Puente del acueducto de Segovia.}
	\label{fig:fig6}
\end{figure}

%%%%%%%%%%%%%%%%%%%%%%%%%%%%%%%%%%%%%%%%%%%%%%%%%%%%%%%%%%%%%%%%%%%%%%%%%%%

\subsection{Retroceso y Reinvenci\'on de la Mamposter\'{\i}a}

El desarrollo de la civilizaci\'on europea se detuvo a lo largo de la edad media, en el aspecto constructivo no fue diferente ya que se dejan de fabricar ladrillos por varios siglos y desaparece la tecnolog\'{\i}a del cemento y el concreto.

Es hasta el siglo XII cuando se reinventa la mamposter\'{\i}a dado que los arcos sumerios y romano de medio punto se sustituyen por el arco g\'otico y la b\'oveda de crucer\'{\i}a que permitieron cubrir grandes claros. Para estos nuevos sistemas se utiliza mamposter\'{\i}a de arcilla o piedra con juntas gruesas de morteros de cal. En Europa occidental se prefiere utilizar la mamposter\'{\i}a para controlar desastrosos incendios. Cerca del a\~no $1500$ a.C. se construye la gran muralla China con una altura de $9$ metros construida con ladrillos de arcilla unidos con mortero de cal.

Con la revoluci\'on industrial se extiende la aplicaci\'on de la mamposter\'{\i}a de ladrillos de arcilla y se mejora la producci\'on mediante el uso del gas como combustible y la producci\'on de hornos m\'as eficientes.

La mamposter\'{\i}a de ladrillo europea se lleva al nuevo mundo realizando un sin n\'umero de edificaciones coloniales.

\subsection{La Mamposter\'{\i}a se Refuerza}

El ingeniero brit\'anico Brunel propone por primera vez la aplicaci\'on de refuerzo de hierro forjado en una chimenea contruida de mamposter\'{\i}a. En $1889$ un ingeniero frances de nombre Paul Cottancin presenta un m\'etodo patentado para la aplicaci\'on de refuerzo en edificios de mamposter\'{\i}a.

\newpage

Los siguientes tipos de muros de mamposter\'{\i}a modernos que se pueden identificar en las Normas T\'ecnicas Complementarias (NTC--$2004$) \cite{NTCII2004}.

\begin{itemize}
	\item Muros diafragma o de relleno.
	\item Muros de mamposter\'{\i}a confinada.
	\item Mamposter\'{\i}a reforzada interiormente.
	\item Mamposter\'{\i}a no reforzada.
	\item Mamposter\'{\i}a de piedras naturales.
\end{itemize}

\begin{figure}[htbp]
	\centering
		\includegraphics[scale=0.08]{fig9.png}
		\includegraphics[scale=0.204]{figm10.png}
	\caption{Construcciones de mamposter\'{\i}a con refuerzo.}
	\label{fig:fig9}
\end{figure}

\subsubsection{Muros de Mamposter\'{\i}a Confinada}

Los muros confinados est\'an rodeados por castillos y dalas  formando un marco r\'{\i}gido, en su construcci\'on se coloca la mamposter\'{\i}a y los elementos de concreto a la par; se deben de cumplir diversos requisitos tanto geom\'etricos como de refuerzo definidos en la secci\'on de las Normas T\'ecnicas Complementarias para Dise\~no y Construcci\'on de Estructuras de Mamposter\'{\i}a \cite{NTCII2004}.

\subsubsection{Muros de Mamposter\'{\i}a Reforzada Interiormente}

Los muros reforzados interiormente se conforman en su mayor\'{\i}a de piezas huecas dentro de las cuales se coloca refuerzo en sentido vertical y horizontal conformando elementos tipo castillo y dalas en su interior, se deben cumplir diversos requisitos tanto geom\'etricos como de refuerzo definidos en la secci\'on de las Normas T\'ecnicas Complementarias para Dise\~no y Construcci\'on de Estructuras de Mamposter\'{\i}a \cite{NTCII2004}.

\subsubsection{Muros de Mamposter\'{\i}a no Reforzada}

Se denomina de esta forma a los muros de mamposter\'{\i}a que no cumplen con los diversos requisitos tanto geom\'etricos como de refuerzo definidos en la secci\'on de las Normas T\'ecnicas Complementarias para Dise\~no y Construcci\'on de Estructuras de Mamposter\'{\i}a \cite{NTCII2004}. Estos muros no deben cumplir funciones estructurales.

\subsubsection{Muros de Mamposter\'{\i}a de Piedras Naturales}

Se denomina de esta forma a los muros de mamposter\'{\i}a que se componen de piedras naturales labradas, su principal uso es como elementos divisorios en exterior.
	
\begin{figure}[htbp]
	\centering
		\includegraphics[scale=0.65]{figmmamp1.png}
	\caption{Requisitos de los muros de mamposter\'{\i}a confinada \cite{NTC2004}.}
	\label{figm:fig1}
\end{figure}

%%%%%%%%%%%%%%%%%%%%%%%%%%%%%%%%%%%%%%%%%%%%%%%%%%%%%%%%%%%%%%%%%%%%%%%%%%%%%%%%%%%%%%%%%%%%%%%%%%%%%%%%%%%%%%%%%%%%%%%%%%%%%%%%%%%%