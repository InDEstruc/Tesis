%Autobiografia

\chapter*{Ficha autobiogr�fica}
\chaptermark{Ficha autobiogr�fica}
\label{lastpage}

\begin{figure} [htbp]
\centering
\includegraphics[width=30mm]{fp1.pdf}
\label{fig:fp1}
\end{figure}

\begin{center}
\autor

Candidato para el grado de Maestro en Ingenier\'{\i}a\\
con especialidad en Ingenier\'{\i}a Estructural

\uanl

\fic\bigskip
\begin{center}
Tesis:
	
\begin{tabular}{p{11cm}}
	\centering
	\scshape{\large{\titulo}}
\end{tabular}
\end{center}

\newpage

\bigskip
\LARGE Informaci\'on B\'asica
\normalsize

\begin{itemize}
	\item Lugar y Fecha de Nacimiento: Fresnillo, Zacatecas, M\'exico;  $28$ de mayo de $1989$. 
	\item Nacionalidad: Mexicana.
	\item Estado Civil: Soltero.
	\item Direcci\'on Local: Calle Rio Orinoco $#229$ C, Colonia del Valle,  San Pedro Garza Garc\'{\i}a, Nuevo Le\'on, M\'exico. 
	\item Direcci\'on For\'anea: Calle Congreso del Estado $#14$, Fracc. Gonzales Ortega,  Fresnillo, Zacatecas, M�xico. 
	\item Tel\'efono de Contacto: $811-311-01-75$.
	\item P�gina Web de Trabajo Compartido: https://github.com/InDEstruc 
	\item 	Correo Electr\'onico: ing.emrm@gmail.com. 
\end{itemize}

\LARGE Formaci\'on Acad\'emica B\'asica
\normalsize

\begin{itemize}
	\item INSTRUCCI\'ON PRIMARIA, $1995-2001$: Primaria Jos\'e Mar\'{\i}a Morelos en la Cd. de Fresnillo, Zac.
	\item INSTRUCCI\'ON SECUNDARIA, $2001-2004$: Secundaria L\'azaro C\'ardenas del R\'{\i}o en la Cd. de Fresnillo, Zac.
	\item INSTRUCCI\'ON DE BACHILLERATO, $2004-2007$: Bachillerato F\'{\i}sico-Matem\'atico en la Escuela Preparatoria N\'umero Tres de la Universidad Aut\'onoma de Zacatecas en la Cd. de Fresnillo, Zac.
\end{itemize}

\newpage

\LARGE Estudios Profesionales
\normalsize

\begin{itemize}
	\item LICENCIATURA EN INGENIER\'IA CIVIL, $2007-2012$: Universidad Aut\'onoma de Zacatecas en la ciudad de Zacatecas, Zac. Graduado con honor por titulaci\'on con el promedio m\'as alto de la generaci\'on.
\end{itemize}

\LARGE Formaci\'on Complementaria
\normalsize

\begin{itemize}
	\item UNIVERSIDAD AUT\'ONOMA DE ZACATECAS. Curso de AUTOCAD-B\'asico. $14$ de Febrero de $2009$ al $14$ de Marzo de $2009$.
	\item UNIVERSIDAD AUT\'ONOMA DE ZACATECAS. Curso de AUTOCAD-Intermedio. $21$ de Marzo de $2009$ al $9$ de Mayo de $2009$.
	\item UNIVERSIDAD AUT\'ONOMA DE ZACATECAS. Curso de AUTOCAD-Avanzado-$3$-D. $23$ de Mayo de $2009$ al $20$ de Junio de $2009$.
	\item ASOCIACI\'ON NACIONAL DE ESTUDIANTES DE INGENIER\'IA CIVIL A.C. (ANEIC). Participaci\'on en la XXV OLIMPIANEIC realizada en la Cd. De Culiac\'an Sinaloa en la prueba de Matem\'aticas en la Ingenier\'{\i}a. $30$ de Abril de $2009$ al $3$ de Mayo de $2009$.
	\item UNIVERSIDAD AUT\'ONOMA DE ZACATECAS. Curso de CIVIL-CAD. $17$ de Marzo de $2012$ al $28$ de Abril de $2012$.
	\item INSTITUTO MEXICANO DEL CEMENTO Y EL CONCRETO. (IMCyC). Curso de T\'ecnico de Pruebas al Concreto en Obra Grado I. $28$ y $29$ de Mayo de $2012$.
	\item FORO INTERNACIONAL DE CONCRETO. Tecnolog\'{\i}a, Concreto y Desarrollo. Del $29$ al $30$ de Mayo de $2012$.
	\item CONGRESO INTERNACIONAL DE INGENIER\'IA CIVIL, El cambio clim\'atico como nuevo reto a la ingenier\'{\i}a. Asistencia del $21$ al $23$ de Noviembre de $2012$, Monterrey, Nuevo Le\'on.
	\item SOCIEDAD MEXICANA DE INGENIER\'IA ESTRUCTURAL (SMIE). Curso de Edificaciones de Mamposter\'{\i}a. $03$ de Octubre de $2013$ en Puebla, Puebla.
	\item SOCIEDAD MEXICANA DE INGENIER\'IA ESTRUCTURAL (SMIE). Octavo Simposio Nacional sobre Ingenier\'{\i}a Estructural en la Vivienda. Del $04$ al $05$ de Octubre de $2013$ en Puebla, Puebla.
	\item XIX CONGRESO NACIONAL DE INGENIER\'IA S\'ISMICA. Curso taller del Manual de Obras Civiles de la CFE de Dise\~no por Sismo. Revisi\'on $2013$. Del $6$ al $9$ de noviembre de $2013$,  Boca del R\'{\i}o, Veracruz.
	\item XIX CONGRESO NACIONAL DE INGENIER\'IA S\'ISMICA. Asistencia al Congreso. Del $6$ al $9$ de noviembre de $2013$,  Boca del R\'{\i}o, Veracruz.
\end{itemize}

\LARGE Conocimientos y Habilidades T\'ecnicas
\normalsize

\begin{itemize}
	\item Certificaci\'on de T\'ecnico de Pruebas de Concreto en Obra Grado I por el ACI (American Concrete Institute). Obtenido a trav\'es del IMCyC (Instituto Mexicano del Cemento y el Concreto).
	\item Conocimiento Intermedio de Ingles. Mejor capacidad en lectura y comprensi\'on.
	\item Conocimiento Avanzado de Paqueter\'{\i}a de Office. Word, PowerPoint y Excel. 
	\item Conocimiento Avanzado de Paquetes Latex.
	\item Habilidades de Programaci\'on Intermedias en lenguajes TCL, PYTHON y FORTRAN.
	\item Conocimiento Avanzado de AutoCAD. Principalmente versi\'on $2013$ y anteriores. 
	\item Conocimiento Intermedio de CivilCAD. 
	\item Conocimientos de An\'alisis y Modelado de Elemento Finito.
	\item Manejo de SAP $2000$. Software de an\'alisis y dise\~no estructural, principalmente la versi\'on $14.00$.
	\item Manejo de ECOgcW. Software de an\'alisis y dise\~no estructural.
	\item Manejo de OPENSEES. Versi\'on actual del software de An\'alisis No Lineal por micro modelos (elementos finitos por fibras). 
	\item Manejo de la familia de programas Response-$2000$. Software de An\'alisis No Lineal de elementos individuales de concreto.
	\item Manejo de MATLAB. Principalmente la versi\'on $R2008b$ y anteriores; programaci\'on y manipulaci\'on de informaci\'on.
	\item Manejo de MATHCAD. Principalmente la versi\'on $14$ y $15$.
	\item Manejo del Software R. Software de an\'alisis avanzado de estad\'{\i}sticas.
	\item Manejo del Software Estad\'{\i}stico Instat. Versi\'on de an\'alisis avanzado de estad\'{\i}sticas fenomenol\'ogicas. 
	\item Manejo b\'asico del Software WMS $8.3$. Programa para manejo de recursos hidrol\'ogicos y geo-referenciaci\'on. 
	\item Manejo b\'asico del Software ArcGIS $9.0$. Programa para el manejo de recursos hidrol\'ogicos y geo-referenciaci\'on.  
\end{itemize}

\LARGE Experiencia Laboral y Profesional
\normalsize

\begin{itemize}
	\item Memoria de C�lculo. Remodelaci�n estructural a cargo del Dr. Ra�l Barr�n Corvera.
Propietario: Mario Torres.
Ubicaci�n: Salida a San Ram�n S/N
Esquina con calle Luna
Fraccionamiento Jardines del Sol
Guadalupe, Zac.
Fecha: Marzo de 2011.
	\item Memoria de C�lculo. Proyecto Nave Industrial Bodega Los Astros.
Cliente: Ing. Oscar Rene Mu�oz V�zquez. Fecha: Agosto 2014.
Ubicaci�n: Av. De los Astros esquina con Av. Del centro sur,
Barrio San Pedro en Ciudad Solidaridad
Monterrey, Nuevo Le�n.
	\item Memoria de C�lculo.
Proyecto Cubierta Canchas del Bosque. Fecha: Agosto 2014.
Cliente: Gobierno de C�rdoba Veracruz.
Ubicaci�n: Calle Santa M�nica, Colonia El trebol
C�rdoba, Veracruz.
	\item Memoria de C�lculo.
Proyecto Puente Parque. Fecha: Septiembre 2014. En curso.
Cliente: Gobierno de C�rdoba Veracruz.
Ubicaci�n: Entre Av. 4 y Calle 1 sobre el cauce del r�os San Antonio,
C�rdoba Veracruz.
	\item Memoria de C�lculo.
Proyecto Casa Francisco. Fecha: Septiembre 2014. En curso.
Cliente: Arq. Patricia Barajas de la Rosa.
Ubicaci�n: Calle Ignacio Zaragoza,
Colonia Esparza.
Fresnillo, Zacatecas.
	\item Memoria de C�lculo.
Proyecto Nave Av�cola. Fecha: Septiembre 2014. En curso.
Cliente: Guillermo Mart�nez Carrillo.
Ubicaci�n: Rancho Don Ramiro,
Fresnillo, Zacatecas.
\end{itemize}

\LARGE Intereses Personales
\normalsize

\begin{itemize}
	\item Continuar con el aprendizaje del idioma Ingles y con el aprendizaje de otros idiomas (Alem�n y Frances).
	\item  Pr�ctica de Violonchelo.
	\item  Pr�ctica deportiva de boxeo como peleador amateur.
	\item  Intereses y pr�ctica de la astronom�a.
\end{itemize}

%%%%%%%%%%%%%%%%%%%%%%%%%%%%%%%%%%%%%%%%%%%%%%%%%%%%%%%%%%%%%%%%%%%%%%%%%%%%%%%%%%%%%%%%%%%%%%%%%%%%%%%%%%%%%%%%%%%%%%%%%%%%%%%%%%%%%%%%%%%%%%%%%%%%%%%%%%%%