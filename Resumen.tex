%Resumen

\chapter{Resumen}
\markboth{Resumen}{}

\noindent\autor.

%Pon aqu� el grado
\noindent Candidato para el Grado de Maestro en Ciencias\\
\indent con Especialidad en Ingenier\'{\i}a Estructural.

\noindent\uanl.\\
\noindent\fic.

\noindent T\'{\i}tulo del Estudio:

\begin{center}
\begin{tabular}{p{11cm}}
	\centering
	\scshape{\large{\titulo}}
\end{tabular}
\end{center}\bigskip

\paragraph{Objetivos y m\'etodo de estudio:}

En esta investigaci\'on se desarrollan diversas herramientas y m\'etodos para la obtenci\'on de funciones de confiabilidad s\'{\i}smica para estructuras de concreto reforzado en el espacio; posteriormente se estudian edificios con la finalidad espec\'{\i}fica de evaluar la presencia no planeada de muros de mamposter\'{\i}a en una configuraci\'on de piso suave en planta baja. La hip\'otesis principal es demostrada debido a que la presencia no planeada de los muros de mamposter\'{\i}a es capaz de influenciar en el comportamiento global de la estructura, de manera que los niveles de confiabilidad se ven disminuidos.

\newpage

Se presenta una metodolog\'{\i}a adicional que permite caracterizar el comportamiento espacial en t\'erminos de confiabilidad de manera que es posible encontrar debilidades estructurales que conllevan a un comportamiento torsional indeseable y de esta forma ser\'{\i}a posible realizar propuestas de optimaci\'on estructural. Con el esquema propuesto de confiabilidad espacial a trav\'es de futuras investigaciones ser\'{\i}a posible generar diagramas de interacci\'on para diversos estados de da\~no.

\paragraph{Contribuciones Finales:} Como resultado de la investigaci\'on se resumen las siguientes contribuciones:
\begin{itemize}
	\item [a)] Se presenta una m\'etodolog\'{\i}a de simulaci\'on de incertidumbres inherentes en edificios de concreto reforzado a trav\'es del c\'odigo SIB que permite realizar gran cantidad de estudios posteriores sobre la influencia espec\'{\i}fica de las variables consideradas de manera que es posible realizar an\'alisis de sensibilidad y caracterizar esquemas de falla y comportamiento en gran variedad de configuraciones estructurales. Los resultados de esta parte de la investigaci\'on se presentan a detalle en un art\'{\i}culo resultado de la presente investigaci\'on \cite{SIB2013}.
	\item [b)] Se realiza la simulaci\'on y an\'alisis de un grupo de sistemas tipo ECMD (Estructuras de Concreto con Mamposter\'{\i}a Desligada) modificando el modelo de magnitud y distribuci\'on espacial de la carga viva; espec\'{\i}ficamente los modelos de Mitchel, G. R. y Woodgate, R. W \cite{MG1970} y el de Peir J. y Cornell C. \cite{PC1973}. 
	Los resultados de esta parte de la investigaci\'on se presentan a detalle en un art\'{\i}culo resultado de la presente investigaci\'on \cite{SIBCV2014}.
	\item [c)] Se presenta un esquema de observaci\'on del comportamiento y seguridad estructural a trav\'es de un grupo de an\'alisis espaciales que permiten caracterizar direcciones de carga de gran debilidad o posibles puntos de colapso.	
	\item [d)] Se presenta un macromodelo de tipo puntal no lineal discretizado por fibras  \cite{DC1994} calibrado y puesto en marcha para informaci\'on de mamposter\'{\i}a mexicana \cite{AA2001}.
	\item [e)] Se obtienen las funciones de confiabilidad de estructuras tipo ECML (Estructura de Concreto con Mamposter\'{\i}a Ligada) a partir de estructuras tipo ECMD con el fundamente de una presencia no planeada de la mamposter\'{\i}a.
\end{itemize}

\noindent Firma del asesor interno: \rule{51mm}{0.3pt}

\noindent\hspace{37mm} \asesor

\hspace{37mm} 

\noindent Firma del asesor externo: \rule{51mm}{0.3pt}

\noindent\hspace{37mm} \coasesor

\chapter{Abstract}

In this research different tools and methods for obtaining seismic reliability functions of spatial reinforced concrete structures are developed, subsequently buildings were studied to assessing the presence unplanned of masonry walls in a soft ground floor. The main hypothesis was demostrated because the unintended presence of masonry walls is able to influence the overall behavior of the structure , so that the reliability levels are diminished.
An additional methodology for characterizing the spatial behavior in terms of reliability is presented, so it is possible to find structural weaknesses that lead to an undesirable torsional behavior and would be possible to make proposals for structural optimization. The proposed spatial reliability through future research could generate interaction diagrams for various damage states scheme. Several recommendations for future research related to the topics covered are established. 

%%%%%%%%%%%%%%%%%%%%%%%%%%%%%%%%%%%%%%%%%%%%%%%%%%%%%%%%%%%%%%%%%%%%%%%%%%%%%%%%%%%%%%%%%%%%%%%%%%%%%%%%%%%%%%%%%%%%%%%%%%%%%%%%%%%%%%%%%%%%%%%%%%%%%%%%%%%%