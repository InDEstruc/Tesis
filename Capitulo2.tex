\section{An\'alisis de Confiabilidad}

A continuaci\'on se presentan los resultados finales obtenidos a trav\'es de la modelaci\'on no lineal que permiten obtener las funciones de confiabilidad; estos resultados corresponden a los niveles de desplazamiento generadas en los entrepisos del nivel superior. Los resultados se presentan de acuerdo al esquema propuesto de confiabilidad en el espacio. En el \emph{Apendice \textbf{D}} se presentan resultados complementarios a los presentados a continuaci\'on.

\subsection{An\'alisis en Direcci\'on $+X$.}

En la Figura \ref{fig:rf31} y \ref{fig:rf32} se presentan las gr\'aficas de confiabilidad obtenidas para los sistemas ECMD y ECML respectivamente para el caso de direcci\'on positiva en X. Los resultados se presentan de acuerdo al esquema propuesto de confiabilidad en el espacio.

\begin{figure} [htbp]
\centering
\includegraphics[width=130mm]{GFC3XP1.pdf}
\caption{Gr\'aficas de confiabilidad en direcci\'on $+X$ para la estructura sin muros de mamposter\'{\i}a.}
\label{fig:rf31}
\end{figure}

\begin{figure} [htbp]
\centering
\includegraphics[width=130mm]{GFC3XP1CM.pdf}
\caption{Gr\'aficas de confiabilidad en direcci\'on $+X$ para la estructura con muros de mamposter\'{\i}a.}
\label{fig:rf32}
\end{figure}

\newpage

\subsection{An\'alisis en Direcci\'on -X.}

En la Figura \ref{fig:rf37} y \ref{fig:rf38} se presentan las gr\'aficas de confiabilidad obtenidas para los sistemas ECMD y ECML en la aplicaci\'on de carga en direcci\'on X negativa global. Los resultados se presentan de acuerdo al esquema propuesto de confiabilidad en el espacio.

\begin{figure} [htbp]
\centering
\includegraphics[width=130mm]{GFC3XP2.pdf}
\caption{Gr\'aficas de confiabilidad en direcci\'on $-X$ para la estructura sin muros de mamposter\'{\i}a.}
\label{fig:rf37}
\end{figure}

\begin{figure} [htbp]
\centering
\includegraphics[width=130mm]{GFC3XP2CM.pdf}
\caption{Gr\'aficas de confiabilidad en direcci\'on $-X$ para la estructura con muros de mamposter\'{\i}a.}
\label{fig:rf38}
\end{figure}

\newpage

\subsection{An\'alisis en Direcci\'on +Z.}

En la Figura \ref{fig:rf47} y \ref{fig:rf48} se presentan las gr\'aficas de confiabilidad obtenidas para los sistemas ECMD y ECML en la aplicaci\'on de carga en direcci\'on Z positiva global. Los resultados se presentan de acuerdo al esquema propuesto de confiabilidad en el espacio.

\begin{figure} [htbp]
\centering
\includegraphics[width=130mm]{GFC3ZP3.pdf}
\caption{Gr\'aficas de confiabilidad en direcci\'on $+Z$ para la estructura sin muros de mamposter\'{\i}a.}
\label{fig:rf47}
\end{figure}

\begin{figure} [htbp]
\centering
\includegraphics[width=130mm]{GFC3ZP3CM.pdf}
\caption{Gr\'aficas de confiabilidad en direcci\'on $+Z$ para la estructura con muros de mamposter\'{\i}a.}
\label{fig:rf48}
\end{figure}

\newpage

\subsection{An\'alisis en Direcci\'on -Z.}

En la Figura \ref{fig:rf53} y \ref{fig:rf54} se presentan las gr\'aficas de confiabilidad obtenidas para los sistemas ECMD y ECML respectivamente en la aplicaci\'on de carga en direcci\'on Z negativa global. Los resultados se presentan de acuerdo al esquema propuesto de confiabilidad en el espacio.

\begin{figure} [htbp]
\centering
\includegraphics[width=130mm]{GFC3ZP4.pdf}
\caption{Gr\'aficas de confiabilidad en direcci\'on $-Z$ para la estructura sin muros de mamposter\'{\i}a.}
\label{fig:rf53}
\end{figure}

\begin{figure} [htbp]
\centering
\includegraphics[width=130mm]{GFC3ZP4CM.pdf}
\caption{Gr\'aficas de confiabilidad en direcci\'on $-Z$ para la estructura con muros de mamposter\'{\i}a.}
\label{fig:rf54}
\end{figure}

\newpage

\subsection{An\'alisis en Direcci\'on $+X+Z$.}

En la Figura \ref{fig:rf63} y \ref{fig:rf64} se presentan las gr\'aficas de confiabilidad obtenidas para los sistemas ECMD y ECML respectivamente en la aplicaci\'on de cargas en direcci\'on $+X$ y $+Z$, controlando el pushover en X. Los resultados se presentan de acuerdo al esquema propuesto de confiabilidad en el espacio.

\begin{figure} [htbp]
\centering
\includegraphics[width=130mm]{GFC3XZP5.pdf}
\caption{Gr\'aficas de confiabilidad en direcci\'on $+X+Z$ para la estructura sin muros de mamposter\'{\i}a.}
\label{fig:rf63}
\end{figure}

\begin{figure} [htbp]
\centering
\includegraphics[width=130mm]{GFC3XZP5CM.pdf}
\caption{Gr\'aficas de confiabilidad en direcci\'on $+X+Z$ para la estructura con muros de mamposter\'{\i}a.}
\label{fig:rf64}
\end{figure}

\newpage

\subsection{An\'alisis en Direcci\'on +X-Z.}

En la Figura \ref{fig:rf69} y \ref{fig:rf70} se presentan las gr\'aficas de confiabilidad obtenidas para los sistemas ECMD y ECML respectivamente en la aplicaci\'on de carga en direcci\'on $+X$ y $-Z$, controlando el pushover en X. Los resultados se presentan de acuerdo al esquema propuesto de confiabilidad en el espacio.

\begin{figure} [htbp]
\centering
\includegraphics[width=130mm]{GFC3XZP6.pdf}
\caption{Gr\'aficas de confiabilidad en direcci\'on $+X-Z$ para la estructura sin muros de mamposter\'{\i}a.}
\label{fig:rf69}
\end{figure}

\begin{figure} [htbp]
\centering
\includegraphics[width=130mm]{GFC3XZP6CM.pdf}
\caption{Gr\'aficas de confiabilidad en direcci\'on $+X-Z$ para la estructura con muros de mamposter\'{\i}a.}
\label{fig:rf70}
\end{figure}

\newpage

\subsection{An\'alisis en Direcci\'on -X+Z.}

En la Figura \ref{fig:rf75} y \ref{fig:rf76} se presentan las gr\'aficas de confiabilidad obtenidas para los sistemas ECMD y ECML respectivamente en la aplicaci\'on de carga en direcci\'on $-X$ y $+Z$, controlando el pushover en Z. Los resultados se presentan de acuerdo al esquema propuesto de confiabilidad en el espacio.

\begin{figure} [htbp]
\centering
\includegraphics[width=130mm]{GFC3XZP7.pdf}
\caption{Gr\'aficas de confiabilidad en direcci\'on $-X+Z$ para la estructura sin muros de mamposter\'{\i}a.}
\label{fig:rf75}
\end{figure}

\begin{figure} [htbp]
\centering
\includegraphics[width=130mm]{GFC3XZP7CM.pdf}
\caption{Gr\'aficas de confiabilidad en direcci\'on $-X+Z$ para la estructura con muros de mamposter\'{\i}a.}
\label{fig:rf76}
\end{figure}

\newpage

\subsection{An\'alisis en Direcci\'on -X-Z.}

En la Figura \ref{fig:rf81} y \ref{fig:rf82} se presentan las gr\'aficas de confiabilidad obtenidas para los sistemas ECMD y ECML respectivamente en la aplicaci\'on de carga en direcci\'on $-X$ y $-Z$, controlando el pushover en Z. Los resultados se presentan de acuerdo al esquema propuesto de confiabilidad en el espacio.

\begin{figure} [htbp]
\centering
\includegraphics[width=130mm]{GFC3XZP8.pdf}
\caption{Gr\'aficas de confiabilidad en direcci\'on $-X-Z$ para la estructura sin muros de mamposter\'{\i}a.}
\label{fig:rf81}
\end{figure}

\begin{figure} [htbp]
\centering
\includegraphics[width=130mm]{GFC3XZP8CM.pdf}
\caption{Gr\'aficas de confiabilidad en direcci\'on $-X-Z$ para la estructura con muros de mamposter\'{\i}a.}
\label{fig:rf82}
\end{figure}

\newpage

\subsection{Observaciones Generales}

De los anteriores resultados se observa que es posible establecer relaciones de da\~no en el espacio. Resulta conservador seleccionar los valores m\'{\i}nimos del \'{\i}ndice $\beta$ que corresponden en el espacio simulado a los valores obtenidos en la direcci\'on $+Z$; lo anterior posiblemente reside en: la menor intensidad de las aceleraciones registradas en direcci\'on $N00E$ y por otro lado en el truncamiento de las funciones de distribuci\'on debido a la falta de simulaciones. 

\begin{figure} [htbp]
\centering
\includegraphics[width=160mm]{GFCFINAL1.pdf}
\caption{Gr\'afica de confiabilidad espacial m\'{\i}nima comparativa de estructuras ECMD y ECML.}
\label{fig:rf83}
\end{figure}

En la Figura \ref{fig:rf83} se reproducen los valores del \'{\i}ndice $\beta$ considerados, como puede observarse las estructuras sin muros tipo ECMD resultan de menor confiabilidad para rangos de intensidad normalizada que mantienen a la estructura en el rango el\'astico, posteriormente conforme se incrementa la intensidad y la estructura entra en un estado de deformaci\'on considerable y la confiabilidad de las estructuras ECML resulta menor, esto representa la generaci\'on del fen\'omeno de piso suave en planta baja.

\begin{figure} [htbp]
\centering
\includegraphics[width=160mm]{GFCFINAL3.pdf}
\caption{Gr\'afica de diferencias $\beta$ de estructuras tipo ECMD y ECML.}
\label{fig:rf84}
\end{figure}

\newpage

\begin{figure} [htbp]
\centering
\includegraphics[width=160mm]{GFCFINAL2.pdf}
\caption{Kernel Gaussiano de la diferencia del \'{\i}ndice $\beta$ entre estructuras ECMD y ECML.}
\label{fig:rf85}
\end{figure}

En la Figura \ref{fig:rf84} se muestran los valores absolutos de la diferencia num\'erica entre el \'{\i}ndice $\beta$ de estructuras tipo ECMD y ECML, como puede observarse el comportamiento obedece la tendencia de distinci\'on entre la etapa el\'astica e inel\'astica y el valor de la intensidad en que cambia la tendencia de diferencias es de $q\approx 0.80$. En la Figura \ref{fig:rf85} se presentan el kernel gaussiano de los valores absolutos de las diferencias entre estructuras ECMD y ECML en el cual puede observarse que la tendencia de diferencia es de $1.0$ unidades del \'{\i}ndice $\beta$.

 Finalmente de un proceso comparativo entre investigaciones anteriores \cite{D2008, Put2010, PE2012} semejantes y los resultados presentados, resultan las siguientes observaciones:

\begin{itemize}
	\item Los niveles de confiabilidad obtenidos en la presente investigaci\'on resultan menores a los obtenidos en las investigaciones comparativas \cite{D2008,Put2010} respecto a los niveles de intensidad normalizada.
	\item La intensidad normalizada requerida para obtener valores negativos del \'{\i}ndice $\beta$ en la presente investigaci\'on resulta de valores alrededor de $q=1.10$, por otro lado los valores obtenidos en las investigaciones comparativas resultan alrededor de $q=5.0$ para estructuras tipo ECMD y de $\beta=6.0$ para estructura tipo ECML con \emph{piso suave} en planta baja \cite{Put2010}, en estas investigaciones no se obtienen valores negativos del \'{\i}ndice $\beta$.
	\item Los valores de la intensidad normalizada para los que se logra la convergencia entre las curvas de estructuras ECMD y ECML es de $q \approx 0.8$ para $\beta \approx 2.0$; en investigaciones comparativas por otro lado se obtienen valores de $q \approx 0.8$ para $\beta \approx 3.5$ \cite{Put2010}.	
	\item En la presente investigaci\'on es posible visualizar los efectos tridimensionales de manera que se obtiene el comportamiento general del sistema.  Resulta claro que la confiabilidad var\'{\i}a espacialmente y siendo una estructura con alta irregularidad esta metodolog\'{\i}a resulta adecuada para obtener los niveles de confiabilidad.
\end{itemize}

Es importante observar que se ha establecido un marco te\'orico que permite observar los efectos de irregularidad que pueden repercutir en fen\'omenos torsionales a trav\'es de la medida del \'{\i}ndice $\beta$ en un diagrama de interacci\'on tridimensional, sin embargo, para los casos estudiados en esta investigaci\'on no fue posible observar diferencias significativas. Es necesario resaltar que los valores de los desplazamientos y el cortante basal corresponden a valores absolutos resultantes y no valores unidimensionales como en otras investigaciones \cite{D2008, Put2010, PE2012}.

Es necesario realizar un mayor n\'umero de simulaciones, las cuales en esta investigaci\'on no fueron posibles de realizar debido a que fue necesario realizar las diversas herramientas de an\'alisis previo a las simulaciones, se recomienda un m\'{\i}nimo de $200$ simulaciones para obtener valores concluyentes sin que se trunquen las distribuciones consideradas.

%%%%%%%%%%%%%%%%%%%%%%%%%%%%%%%%%%%%%%%%%%%%%%%%%%%%%%%%%%%%%%%%%%%%%%%%%%%%%%%%%%%%%%%%%%%%%%%%%%%%%%%%%%%%%%%%%%%%%%%%%%%%%%%%%%%%%%%%%%%%%%%%%%%%%%%%%%%%