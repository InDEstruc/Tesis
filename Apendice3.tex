%%%%%%%%%%%%%%%%%%%%%%%%%%%%%%%%%%%%%%%%%%%%%%%%%%%%%%%%%%%%%%%%%%%%%%%%%%%%%%%%%%%%%%%%%%%%%%%%%%%%%%%%%%%%%%%%%%%%%%%%%%%%%%%%%%%%%%%%%%%%%%%%%%%%%%%%%%%%

\chapter{Resultados del Modelado No Lineal}

\section{Modelado No Lineal de la Mamposter\'{\i}a.}

A continuaci\'on se presentan diversos resultados complementarios de la secuencia de eliminaci\'on de los muros de mamposter\'{\i}a; para lo anterior se sigue la secuencia de estudios de la metodolog\'{'i}a propuesta de confiabilidad.

\begin{figure} [htbp]
\includegraphics[width=150mm]{SEP1SM6.pdf}
\includegraphics[width=150mm]{SEP2SM6.pdf}
\caption{Proceso de remoci\'on de mamposter\'{\i}a ante Pushover.}
\label{fig:apu1}
\end{figure}

\begin{figure} [htbp]
\includegraphics[width=150mm]{SEP3SM6.pdf}
\includegraphics[width=150mm]{SEP4SM6.pdf}
\caption{Proceso de remoci\'on de mamposter\'{\i}a ante Pushover.}
\label{fig:apu2}
\end{figure}

\begin{figure} [htbp]
\includegraphics[width=150mm]{SEP5SM6.pdf}
\includegraphics[width=150mm]{SEP6SM6.pdf}
\caption{Proceso de remoci\'on de mamposter\'{\i}a ante Pushover.}
\label{fig:apu3}
\end{figure}

\begin{figure} [htbp]
\includegraphics[width=150mm]{SEP7SM6.pdf}
\includegraphics[width=150mm]{SEP8SM6.pdf}
\caption{Proceso de remoci\'on de mamposter\'{\i}a ante Pushover.}
\label{fig:apu4}
\end{figure}

\begin{figure} [htbp]
\includegraphics[width=150mm]{SES1.pdf}
\includegraphics[width=150mm]{SES2.pdf}
\caption{Comportamiento dentro y fuera del plano del modelo de mamposter\'{\i}a ante Sismos.}
\label{fig:apu5}
\end{figure}

\begin{figure} [htbp]
\includegraphics[width=150mm]{SES3.pdf}
\includegraphics[width=150mm]{SES4.pdf}
\caption{Comportamiento dentro y fuera del plano del modelo de mamposter\'{\i}a ante Sismos.}
\label{fig:apu6}
\end{figure}

\newpage

\section{An\'alisis de Confiabilidad}

A continuaci\'on se presentan algunos de los resultados obtenidos a trav\'es de la modelaci\'on no lineal complementarios a los del \emph{Cap\'{\i}tulo \textbf{6}}. Estos resultados permiten obtener las funciones de confiabilidad y corresponden a los niveles de desplazamientos absolutos generados en los entrepisos del nivel superior.

\subsection{An\'alisis en Direcci\'on $+X$.}

A continuaci\'on se presentan los principales resultados de la modelaci\'on no lineal de los edificios espec\'{\i}ficamente en la aplicaci\'on del Pushover en direcci\'on X positiva global.

\begin{figure} [htbp]
\includegraphics[width=150mm]{P1SMN5.pdf}
\caption{Pushover de los sistemas con propiedades simuladas sin muros en la direcci\'on $+X$.}
\label{fig:apu7}
\end{figure}

\begin{figure} [htbp]
\includegraphics[width=150mm]{P1ZSMN5.pdf}
\caption{Pushover de los sistemas con propiedades simuladas sin muros en la direcci\'on $+X$.}
\label{fig:apu8}
\end{figure}

\begin{figure} [htbp]
\includegraphics[width=150mm]{P1CMN5.pdf}
\caption{Pushover de los sistemas con propiedades simuladas con muros en la direcci\'on $+X$.}
\label{fig:apu9}
\end{figure}

\begin{figure} [htbp]
\includegraphics[width=150mm]{P1ZCMN5.pdf}
\caption{Pushover de los sistemas con propiedades simuladas con muros en la direcci\'on $+X$.}
\label{fig:apu10}
\end{figure}

En la Figura \ref{fig:apu11} a \ref{fig:apu14} se presenta una de las simulaciones sometida a la componente horizontal X de uno de los sismos considerados. Para cada uno de los pushover en direcci\'on X, sea positiva o negativa se considera la $K_{sec}$ de dicha componente s\'{\i}smica. 

\begin{figure} [htbp]
\includegraphics[width=150mm]{DNL1X.pdf}
\caption{Gr\'afica de un an\'alisis din\'amico no lineal de un sistema con propiedades simuladas sin muros en la direcci\'on $+X$.}
\label{fig:apu11}
\end{figure}

\begin{figure} [htbp]
\includegraphics[width=150mm]{DNL2X.pdf}
\caption{Gr\'afica de un an\'alisis din\'amico no lineal de un sistema con propiedades simuladas sin muros en la direcci\'on $+X$.}
\label{fig:apu12}
\end{figure}

\begin{figure} [htbp]
\includegraphics[width=150mm]{DNL1XCM.pdf}
\caption{Gr\'afica de un an\'alisis din\'amico no lineal de un sistema con propiedades simuladas con muros en la direcci\'on $+X$.}
\label{fig:apu13}
\end{figure}

\begin{figure} [htbp]
\includegraphics[width=150mm]{DNL2XCM.pdf}
\caption{Gr\'afica de un an\'alisis din\'amico no lineal de un sistema con propiedades simuladas con muros en la direcci\'on $+X$.}
\label{fig:apu14}
\end{figure}

En la Figura \ref{fig:apu15} y \ref{fig:apu16} se presentan las gr\'aficas de confiabilidad obtenidas para los sistemas ECMD y ECML respectivamente.

\begin{figure} [htbp]
\centering
\includegraphics[width=140mm]{GFC1XP1.pdf}
\includegraphics[width=140mm]{GFC2XP1.pdf}
\includegraphics[width=140mm]{GFC3XP1.pdf}
\caption{Gr\'aficas de confiabilidad en direcci\'on $+X$ para la estructura sin muros de mamposter\'{\i}a.}
\label{fig:apu15}
\end{figure}

\begin{figure} [htbp]
\centering
\includegraphics[width=140mm]{GFC1XP1CM.pdf}
\includegraphics[width=140mm]{GFC2XP1CM.pdf}
\includegraphics[width=140mm]{GFC3XP1CM.pdf}
\caption{Gr\'aficas de confiabilidad en direcci\'on $+X$ para la estructura con muros de mamposter\'{\i}a.}
\label{fig:apu16}
\end{figure}

\paragraph{}

\newpage

\subsection{An\'alisis en Direcci\'on -X.}

A continuaci\'on se presentan los principales resultados de la modelaci\'on no lineal de los edificios espec\'{\i}ficamente en la aplicaci\'on del Pushover en direcci\'on X negativa global.

\begin{figure} [htbp]
\includegraphics[width=150mm]{P2SMN5.pdf}
\caption{Pushover de los sistemas con propiedades simuladas sin muros en la direcci\'on $-X$.}
\label{fig:apu17}
\end{figure}

\begin{figure} [htbp]
\includegraphics[width=150mm]{P2ZSMN5.pdf}
\caption{Pushover de los sistemas con propiedades simuladas sin muros en la direcci\'on $-X$.}
\label{fig:apu18}
\end{figure}

\begin{figure} [htbp]
\includegraphics[width=150mm]{P2CMN5.pdf}
\caption{Pushover de los sistemas con propiedades simuladas con muros en la direcci\'on $-X$.}
\label{fig:apu19}
\end{figure}

\begin{figure} [htbp]
\includegraphics[width=150mm]{P2ZCMN5.pdf}
\caption{Pushover de los sistemas con propiedades simuladas con muros en la direcci\'on $-X$.}
\label{fig:apu20}
\end{figure}

En la Figura \ref{fig:apu21} y \ref{fig:apu22} se presentan las gr\'aficas de confiabilidad obtenidas para los sistemas ECMD y ECML respectivamente.

\newpage

\begin{figure} [htbp]
\centering
\includegraphics[width=140mm]{GFC1XP2.pdf}
\includegraphics[width=140mm]{GFC2XP2.pdf}
\includegraphics[width=140mm]{GFC3XP2.pdf}
\caption{Gr\'aficas de confiabilidad en direcci\'on $-X$ para la estructura sin muros de mamposter\'{\i}a.}
\label{fig:apu21}
\end{figure}

\begin{figure} [htbp]
\centering
\includegraphics[width=140mm]{GFC1XP2CM.pdf}
\includegraphics[width=140mm]{GFC2XP2CM.pdf}
\includegraphics[width=140mm]{GFC3XP2CM.pdf}
\caption{Gr\'aficas de confiabilidad en direcci\'on $-X$ para la estructura con muros de mamposter\'{\i}a.}
\label{fig:apu22}
\end{figure}

\paragraph{}

\newpage

\subsection{An\'alisis en Direcci\'on +Z.}

A continuaci\'on se presentan los principales resultados de la modelaci\'on no lineal de los edificios espec\'{\i}ficamente en la aplicaci\'on del Pushover en direcci\'on Z positiva global.

\begin{figure} [htbp]
\includegraphics[width=150mm]{P3SMN5.pdf}
\caption{Pushover de los sistemas con propiedades simuladas sin muros en la direcci\'on $+Z$.}
\label{fig:apu23}
\end{figure}

\begin{figure} [htbp]
\includegraphics[width=150mm]{P3XSMN5.pdf}
\caption{Pushover de los sistemas con propiedades simuladas sin muros en la direcci\'on $+Z$.}
\label{fig:apu24}
\end{figure}

\begin{figure} [htbp]
\includegraphics[width=150mm]{P3CMN5.pdf}
\caption{Pushover de los sistemas con propiedades simuladas con muros en la direcci\'on $+Z$.}
\label{fig:apu25}
\end{figure}

\begin{figure} [htbp]
\includegraphics[width=150mm]{P3XCMN5.pdf}
\caption{Pushover de los sistemas con propiedades simuladas con muros en la direcci\'on $+Z$.}
\label{fig:apu26}
\end{figure}

En la Figura \ref{fig:apu27} a \ref{fig:apu30} se presenta una de las simulaciones sometida a la componente horizontal Z de uno de los sismos considerados. Para cada uno de los pushover en direcci\'on Z, sea positiva o negativa se considera la $K_{sec}$ de dicha componente s\'{\i}smica. 

\begin{figure} [htbp]
\includegraphics[width=150mm]{DNL1Z.pdf}
\caption{Gr\'afica de un an\'alisis din\'amico no lineal de un sistema con propiedades simuladas sin muros en la direcci\'on $+Z$.}
\label{fig:apu27}
\end{figure}

\begin{figure} [htbp]
\includegraphics[width=150mm]{DNL2Z.pdf}
\caption{Gr\'afica de un an\'alisis din\'amico no lineal de un sistema con propiedades simuladas sin muros en la direcci\'on $+Z$.}
\label{fig:apu28}
\end{figure}

\begin{figure} [htbp]
\includegraphics[width=150mm]{DNL1ZCM.pdf}
\caption{Gr\'afica de un an\'alisis din\'amico no lineal de un sistema con propiedades simuladas con muros en la direcci\'on $+Z$.}
\label{fig:apu29}
\end{figure}

\begin{figure} [htbp]
\includegraphics[width=150mm]{DNL2ZCM.pdf}
\caption{Gr\'afica de un an\'alisis din\'amico no lineal de un sistema con propiedades simuladas con muros en la direcci\'on $+Z$.}
\label{fig:apu30}
\end{figure}

En la Figura \ref{fig:apu31} y \ref{fig:apu32} se presentan las gr\'aficas de confiabilidad obtenidas para los sistemas ECMD y ECML respectivamente.

\newpage

\begin{figure} [htbp]
\centering
\includegraphics[width=140mm]{GFC1ZP3.pdf}
\includegraphics[width=140mm]{GFC2ZP3.pdf}
\includegraphics[width=140mm]{GFC3ZP3.pdf}
\caption{Gr\'aficas de confiabilidad en direcci\'on $+Z$ para la estructura sin muros de mamposter\'{\i}a.}
\label{fig:apu31}
\end{figure}

\begin{figure} [htbp]
\centering
\includegraphics[width=140mm]{GFC1ZP3CM.pdf}
\includegraphics[width=140mm]{GFC2ZP3CM.pdf}
\includegraphics[width=140mm]{GFC3ZP3CM.pdf}
\caption{Gr\'aficas de confiabilidad en direcci\'on $+Z$ para la estructura con muros de mamposter\'{\i}a.}
\label{fig:apu32}
\end{figure}

\paragraph{}

\newpage

\subsection{An\'alisis en Direcci\'on -Z.}

A continuaci\'on se presentan los principales resultados de la modelaci\'on no lineal de los edificios espec\'{\i}ficamente en la aplicaci\'on del Pushover en direcci\'on Z negativa global.

\begin{figure} [htbp]
\includegraphics[width=150mm]{P4SMN5.pdf}
\caption{Pushover de los sistemas con propiedades simuladas con muros en la direcci\'on $-Z$.}
\label{fig:apu33}
\end{figure}

\begin{figure} [htbp]
\includegraphics[width=150mm]{P4XSMN5.pdf}
\caption{Pushover de los sistemas con propiedades simuladas con muros en la direcci\'on $-Z$.}
\label{fig:apu34}
\end{figure}

\begin{figure} [htbp]
\includegraphics[width=150mm]{P4CMN5.pdf}
\caption{Pushover de los sistemas con propiedades simuladas con muros en la direcci\'on $-Z$.}
\label{fig:apu35}
\end{figure}

\begin{figure} [htbp]
\includegraphics[width=150mm]{P4XCMN5.pdf}
\caption{Pushover de los sistemas con propiedades simuladas con muros en la direcci\'on $-Z$.}
\label{fig:apu36}
\end{figure}

En la Figura \ref{fig:apu37} y \ref{fig:apu38} se presentan las gr\'aficas de confiabilidad obtenidas para los sistemas ECMD y ECML respectivamente.

\newpage

\begin{figure} [htbp]
\centering
\includegraphics[width=140mm]{GFC1ZP4.pdf}
\includegraphics[width=140mm]{GFC2ZP4.pdf}
\includegraphics[width=140mm]{GFC3ZP4.pdf}
\caption{Gr\'aficas de confiabilidad en direcci\'on $-Z$ para la estructura sin muros de mamposter\'{\i}a.}
\label{fig:apu37}
\end{figure}

\begin{figure} [htbp]
\centering
\includegraphics[width=140mm]{GFC1ZP4CM.pdf}
\includegraphics[width=140mm]{GFC2ZP4CM.pdf}
\includegraphics[width=140mm]{GFC3ZP4CM.pdf}
\caption{Gr\'aficas de confiabilidad en direcci\'on $-Z$ para la estructura con muros de mamposter\'{\i}a.}
\label{fig:apu38}
\end{figure}

\paragraph{}

\newpage

\subsection{An\'alisis en Direcci\'on $+X+Z$.}

A continuaci\'on se presentan los principales resultados de la modelaci\'on no lineal de los edificios espec\'{\i}ficamente en la aplicaci\'on del Pushover en direcci\'on X y Z positiva global.

\begin{figure} [htbp]
\includegraphics[width=150mm]{P5SMXN5.pdf}
\caption{Pushover de los sistemas con propiedades simuladas sin muros en la direcci\'on $+X+Z$.}
\label{fig:apu39}
\end{figure}

\begin{figure} [htbp]
\includegraphics[width=150mm]{P5SMZN5.pdf}
\caption{Pushover de los sistemas con propiedades simuladas sin muros en la direcci\'on $+X+Z$.}
\label{fig:apu40}
\end{figure}

\begin{figure} [htbp]
\includegraphics[width=150mm]{P5CMXN5.pdf}
\caption{Pushover de los sistemas con propiedades simuladas con muros en la direcci\'on $+X+Z$.}
\label{fig:apu41}
\end{figure}

\begin{figure} [htbp]
\includegraphics[width=150mm]{P5CMZN5.pdf}
\caption{Pushover de los sistemas con propiedades simuladas con muros en la direcci\'on $+X+Z$.}
\label{fig:apu42}
\end{figure}

En la Figura \ref{fig:apu43} a \ref{fig:apu46} se presenta una de las simulaciones sometida a las tres componentes de uno de los sismos considerados. Para cada uno de los pushover bidireccionales se considera la $K_{sec}$ de dicha componente s\'{\i}smica. 

\begin{figure} [htbp]
\includegraphics[width=150mm]{DNL1XYZ.pdf}
\caption{Gr\'afica de un an\'alisis din\'amico no lineal de un sistema con propiedades simuladas sin muros en la direcci\'on $+X+Z$.}
\label{fig:apu43}
\end{figure}

\begin{figure} [htbp]
\includegraphics[width=150mm]{DNL2XYZ.pdf}
\caption{Gr\'afica de un an\'alisis din\'amico no lineal de un sistema con propiedades simuladas sin muros en la direcci\'on $+X+Z$.}
\label{fig:apu44}
\end{figure}

\begin{figure} [htbp]
\includegraphics[width=150mm]{DNL1XYZCM.pdf}
\caption{Gr\'afica de un an\'alisis din\'amico no lineal de un sistema con propiedades simuladas con muros en la direcci\'on $+X+Z$.}
\label{fig:apu45}
\end{figure}

\begin{figure} [htbp]
\includegraphics[width=150mm]{DNL2XYZCM.pdf}
\caption{Gr\'afica de un an\'alisis din\'amico no lineal de un sistema con propiedades simuladas con muros en la direcci\'on $+X+Z$.}
\label{fig:apu46}
\end{figure}

En la Figura \ref{fig:apu47} y \ref{fig:apu48} se presentan las gr\'aficas de confiabilidad obtenidas para los sistemas ECMD y ECML respectivamente.

\newpage

\begin{figure} [htbp]
\centering
\includegraphics[width=140mm]{GFC1XZP5.pdf}
\includegraphics[width=140mm]{GFC2XZP5.pdf}
\includegraphics[width=140mm]{GFC3XZP5.pdf}
\caption{Gr\'aficas de confiabilidad en direcci\'on $+X+Z$ para la estructura sin muros de mamposter\'{\i}a.}
\label{fig:apu47}
\end{figure}

\begin{figure} [htbp]
\centering
\includegraphics[width=140mm]{GFC1XZP5CM.pdf}
\includegraphics[width=140mm]{GFC2XZP5CM.pdf}
\includegraphics[width=140mm]{GFC3XZP5CM.pdf}
\caption{Gr\'aficas de confiabilidad en direcci\'on $+X+Z$ para la estructura con muros de mamposter\'{\i}a.}
\label{fig:apu48}
\end{figure}

\paragraph{}

\newpage

\subsection{An\'alisis en Direcci\'on +X-Z.}

A continuaci\'on se presentan los principales resultados de la modelaci\'on no lineal de los edificios espec\'{\i}ficamente en la aplicaci\'on del Pushover en direcci\'on X positiva y Z negativa global.

\begin{figure} [htbp]
\includegraphics[width=150mm]{P6SMXN5.pdf}
\caption{Pushover de los sistemas con propiedades simuladas sin muros en la direcci\'on $+X-Z$.}
\label{fig:apu49}
\end{figure}

\begin{figure} [htbp]
\includegraphics[width=150mm]{P6SMZN5.pdf}
\caption{Pushover de los sistemas con propiedades simuladas sin muros en la direcci\'on $+X-Z$.}
\label{fig:apu50}
\end{figure}

\begin{figure} [htbp]
\includegraphics[width=150mm]{P6CMXN5.pdf}
\caption{Pushover de los sistemas con propiedades simuladas con muros en la direcci\'on $+X-Z$.}
\label{fig:apu51}
\end{figure}

\begin{figure} [htbp]
\includegraphics[width=150mm]{P6CMZN5.pdf}
\caption{Pushover de los sistemas con propiedades simuladas con muros en la direcci\'on $+X-Z$.}
\label{fig:apu52}
\end{figure}

En la Figura \ref{fig:apu53} y \ref{fig:apu54} se presentan las gr\'aficas de confiabilidad obtenidas para los sistemas ECMD y ECML respectivamente.

\newpage

\begin{figure} [htbp]
\centering
\includegraphics[width=140mm]{GFC1XZP6.pdf}
\includegraphics[width=140mm]{GFC2XZP6.pdf}
\includegraphics[width=140mm]{GFC3XZP6.pdf}
\caption{Gr\'aficas de confiabilidad en direcci\'on $+X-Z$ para la estructura sin muros de mamposter\'{\i}a.}
\label{fig:apu53}
\end{figure}

\begin{figure} [htbp]
\centering
\includegraphics[width=140mm]{GFC1XZP6CM.pdf}
\includegraphics[width=140mm]{GFC2XZP6CM.pdf}
\includegraphics[width=140mm]{GFC3XZP6CM.pdf}
\caption{Gr\'aficas de confiabilidad en direcci\'on $+X-Z$ para la estructura con muros de mamposter\'{\i}a.}
\label{fig:apu54}
\end{figure}

\paragraph{}

\newpage

\subsection{An\'alisis en Direcci\'on -X+Z.}

A continuaci\'on se presentan los principales resultados de la modelaci\'on no lineal de los edificios espec\'{\i}ficamente en la aplicaci\'on del Pushover en direcci\'on X negativa y Z positiva global.

\begin{figure} [htbp]
\includegraphics[width=150mm]{P7SMXN5.pdf}
\caption{Pushover de los sistemas con propiedades simuladas sin muros en la direcci\'on $-X+Z$.}
\label{fig:apu55}
\end{figure}

\begin{figure} [htbp]
\includegraphics[width=150mm]{P7SMZN5.pdf}
\caption{Pushover de los sistemas con propiedades simuladas sin muros en la direcci\'on $-X+Z$.}
\label{fig:apu56}
\end{figure}

\begin{figure} [htbp]
\includegraphics[width=150mm]{P7CMXN5.pdf}
\caption{Pushover de los sistemas con propiedades simuladas con muros en la direcci\'on $-X+Z$.}
\label{fig:apu57}
\end{figure}

\begin{figure} [htbp]
\includegraphics[width=150mm]{P7CMZN5.pdf}
\caption{Pushover de los sistemas con propiedades simuladas con muros en la direcci\'on $-X+Z$.}
\label{fig:apu58}
\end{figure}

En la Figura \ref{fig:apu59} y \ref{fig:apu60} se presentan las gr\'aficas de confiabilidad obtenidas para los sistemas ECMD y ECML respectivamente.

\newpage

\begin{figure} [htbp]
\centering
\includegraphics[width=140mm]{GFC1XZP7.pdf}
\includegraphics[width=140mm]{GFC2XZP7.pdf}
\includegraphics[width=140mm]{GFC3XZP7.pdf}
\caption{Gr\'aficas de confiabilidad en direcci\'on $-X+Z$ para la estructura sin muros de mamposter\'{\i}a.}
\label{fig:apu59}
\end{figure}

\begin{figure} [htbp]
\centering
\includegraphics[width=140mm]{GFC1XZP7CM.pdf}
\includegraphics[width=140mm]{GFC2XZP7CM.pdf}
\includegraphics[width=140mm]{GFC3XZP7CM.pdf}
\caption{Gr\'aficas de confiabilidad en direcci\'on $-X+Z$ para la estructura con muros de mamposter\'{\i}a.}
\label{fig:apu60}
\end{figure}

\paragraph{}

\newpage

\subsection{An\'alisis en Direcci\'on -X-Z.}

A continuaci\'on se presentan los principales resultados de la modelaci\'on no lineal de los edificios espec\'{\i}ficamente en la aplicaci\'on del Pushover en direcci\'on X negativa y Z negativa global.

\begin{figure} [htbp]
\includegraphics[width=150mm]{P8SMXN5.pdf}
\caption{Pushover de los sistemas con propiedades simuladas sin muros en la direcci\'on $-X-Z$.}
\label{fig:apu61}
\end{figure}

\begin{figure} [htbp]
\includegraphics[width=150mm]{P8SMZN5.pdf}
\caption{Pushover de los sistemas con propiedades simuladas sin muros en la direcci\'on $-X-Z$.}
\label{fig:apu62}
\end{figure}

\begin{figure} [htbp]
\includegraphics[width=150mm]{P8CMXN5.pdf}
\caption{Pushover de los sistemas con propiedades simuladas con muros en la direcci\'on $-X-Z$.}
\label{fig:apu63}
\end{figure}

\begin{figure} [htbp]
\includegraphics[width=150mm]{P8CMZN5.pdf}
\caption{Pushover de los sistemas con propiedades simuladas con muros en la direcci\'on $-X-Z$.}
\label{fig:apu64}
\end{figure}

En la Figura \ref{fig:apu65} y \ref{fig:apu66} se presentan las gr\'aficas de confiabilidad obtenidas para los sistemas ECMD y ECML respectivamente.

\newpage

\begin{figure} [htbp]
\centering
\includegraphics[width=140mm]{GFC1XZP8.pdf}
\includegraphics[width=140mm]{GFC2XZP8.pdf}
\includegraphics[width=140mm]{GFC3XZP8.pdf}
\caption{Gr\'aficas de confiabilidad en direcci\'on $-X-Z$ para la estructura sin muros de mamposter\'{\i}a.}
\label{fig:apu65}
\end{figure}

\begin{figure} [htbp]
\centering
\includegraphics[width=140mm]{GFC1XZP8CM.pdf}
\includegraphics[width=140mm]{GFC2XZP8CM.pdf}
\includegraphics[width=140mm]{GFC3XZP8CM.pdf}
\caption{Gr\'aficas de confiabilidad en direcci\'on $-X-Z$ para la estructura con muros de mamposter\'{\i}a.}
\label{fig:apu66}
\end{figure}

Hasta este punto se presentan los resultados generales y complementarios del \emph{Cap\'{\i}tulo \textbf{6}}.

\newpage

\section{Comportamiento en T\'erminos de Desplazamientos}

A continuaci\'on se presentan algunos de los resultados obtenidos a trav\'es de la modelaci\'on est\'atica no lineal; estos resultados corresponden a los niveles de desplazamiento y rotaci\'on asi como de las distorsiones generadas en los entrepisos.

\begin{figure} [htbp]
\includegraphics[width=150mm]{GDS1X.pdf}
\includegraphics[width=150mm]{GDS1Z.pdf}
\caption{Comportamiento din\'amico ante sismos de la simulaci\'on $1$.}
\label{fig:apu67}
\end{figure}

\begin{figure} [htbp]
\includegraphics[width=150mm]{GDS2X.pdf}
\includegraphics[width=150mm]{GDS2Z.pdf}
\caption{Comportamiento din\'amico ante sismos de la simulaci\'on $2$.}
\label{fig:apu68}
\end{figure}

\begin{figure} [htbp]
\includegraphics[width=150mm]{GDS3X.pdf}
\includegraphics[width=150mm]{GDS3Z.pdf}
\caption{Comportamiento din\'amico ante sismos de la simulaci\'on $3$.}
\label{fig:apu69}
\end{figure}

\begin{figure} [htbp]
\includegraphics[width=150mm]{GDS4X.pdf}
\includegraphics[width=150mm]{GDS4Z.pdf}
\caption{Comportamiento din\'amico ante sismos de la simulaci\'on $4$.}
\label{fig:apu70}
\end{figure}


\begin{figure} [htbp]
\includegraphics[width=150mm]{GDS5X.pdf}
\includegraphics[width=150mm]{GDS5Z.pdf}
\caption{Comportamiento din\'amico ante sismos de la simulaci\'on $5$.}
\label{fig:apu71}
\end{figure}

\begin{figure} [htbp]
\includegraphics[width=150mm]{GDS6X.pdf}
\includegraphics[width=150mm]{GDS6Z.pdf}
\caption{Comportamiento din\'amico ante sismos de la simulaci\'on $6$.}
\label{fig:apu72}
\end{figure}

\begin{figure} [htbp]
\includegraphics[width=150mm]{GDS7X.pdf}
\includegraphics[width=150mm]{GDS7Z.pdf}
\caption{Comportamiento din\'amico ante sismos de la simulaci\'on $7$.}
\label{fig:apu73}
\end{figure}

\begin{figure} [htbp]
\includegraphics[width=150mm]{GDS8X.pdf}
\includegraphics[width=150mm]{GDS8Z.pdf}
\caption{Comportamiento din\'amico ante sismos de la simulaci\'on $8$.}
\label{fig:apu74}
\end{figure}

\begin{figure} [htbp]
\includegraphics[width=150mm]{GDS9X.pdf}
\includegraphics[width=150mm]{GDS9Z.pdf}
\caption{Comportamiento din\'amico ante sismos de la simulaci\'on $9$.}
\label{fig:apu75}
\end{figure}

\begin{figure} [htbp]
\includegraphics[width=150mm]{GDS10X.pdf}
\includegraphics[width=150mm]{GDS10Z.pdf}
\caption{Comportamiento din\'amico ante sismos de la simulaci\'on $9$.}
\label{fig:apu76}
\end{figure}

%%%%%%%%%%%%%%%%%%%%%%%%%%%%%%%%%%%%%%%%%%%%%%%%%%%%%%%%%%%%%%%%%%%%%%%%%%%%%%%%%%%%%%%%%%%%%%%%%%%%%%%%%%%%%%%%%%%%%%%%%%%%%%%%%%%%%%%%%%%%%%%%%%%%%%%%%%%%