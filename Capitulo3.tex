\chapter{ Trabajo Futuro}

\section{Modelado de la Mamposter\'{\i}a}

Debido a los resultados contradictorios obtenidos en esta presente investigaci\'on resulta necesario como primer instancia para el desarrollo de los futuros trabajos estudiar el comportamiento a tensi\'on del puntal de mamposter\'{\i}a del macro modelo considerado
de manera que se asegure el correcto comportamiento. Para lo anterior ser\'a indispensable recurrir a un modelo tridimensional en Opensees cuyas caracter\'{\i}sticas coincidan con pruebas emp\'{\i}ricas de mesa vibradora y cuenten con muros de mamposter\'{\i}a.

\section{Variables y Tipos de Irregularidades}

El esquema de confiabilidad espacial presentado en esta investigaci\'on permite observar diferencias puntuales en el comportamiento de la estructura de manera que es conveniente explorar un mayor n\'umero de casos en los cuales sea posible apreciar los efectos de irregularidad, ya sea por distribuciones asim\'etricas en planta o peso, as\{\i} como de la existencia de n\'ucleos o distribuciones irregulares de muros de mamposter\'{\i}a.

\newpage

A continuaci\'on se presenta un grupo de variables de estudio que comprenden un conjunto adecuado para caracterizar en primer instancia la irregularidad en planta baja debido a la presencia de muros de mamposter\'{\i}a. Cabe aclarar que la configuraci\'on de muros en todos los niveles superiores a la planta baja es la misma; completamente cubierta de muros de relleno de mamposter\'{\i}a, variando unicamente la distribuci\'on en planta baja. 

\begin{itemize}
	\item [a)] Tipo de Irregularidad: 
	\begin{itemize}
		\item [i.] Configuraci\'on de Control tipo ECMD.
				\item [ii.] Configuraci\'on de 	Piso Suave tipo ECML con ausencia de muros en planta baja.
	\end{itemize}
	\item [b)] Factor de comportamiento s\'{\i}smico: 
	\begin{itemize}
		\item [i.] Valor de Q igual a 2.
	\end{itemize}
	\item [c)] Relaci\'on largo-ancho de la base: 
	\begin{itemize}
		\item [i.] Relaci\'on $\frac{a}{b}=1$.
	\end{itemize}
	\item [d)] Tipo de muro: 
	\begin{itemize}
		\item [i.] Sin refuerzo horizontal, de piezas macizas de tab\'{\i}que rojo recocido (el cual se denomina $M2$ \cite{AA2001}).
	\end{itemize}
\end{itemize}

Las Figuras \ref{fig:casosaf1}, \ref{fig:casosaf2}, \ref{fig:casosaf3}, \ref{fig:casosaf4}, \ref{fig:casosaf5} y \ref{fig:casosaf6} presentan de manera visual los posibles tipos de irregularidad y sus correspondientes cortes sobre los ejes ortogonales.  

\begin{figure} [htbp]
\centering
\includegraphics[width=150mm]{b).pdf}
\caption{Configuraci\'on de tipo C.}
\label{fig:casosaf1}
\end{figure}

\begin{figure} [htbp]
\centering
\includegraphics[width=150mm]{bxx).pdf}
\caption{Vistas de la configuraci\'on tipo C.}
\label{fig:casosaf2}
\end{figure}

\begin{figure} [htbp]
\centering
\includegraphics[width=150mm]{byy).pdf}
\caption{Configuraci\'on de tipo Esquina.}
\label{fig:casosaf3}
\end{figure}

\begin{figure} [htbp]
\centering
\includegraphics[width=150mm]{c).pdf}
\caption{Corte X-X de la configuraci\'on de tipo Esquina.}
\label{fig:casosaf4}
\end{figure}

\begin{figure} [htbp]
\centering
\includegraphics[width=150mm]{cxx.pdf}
\caption{Configuraci\'on de esquina.}
\label{fig:casosaf5}
\end{figure}

\begin{figure} [htbp]
\centering
\includegraphics[width=150mm]{cyy.pdf}
\caption{Vistas de la configuraci\'on en esquina.}
\label{fig:casosaf6}
\end{figure}

Como puede observarse existen tres distribuciones generales a estudiar en las cuales las l\'{\i}neas perimetrales muestran cu\'ales contienen muros (l\'{\i}neas con mayor grosor) donde a y b son los anchos de cruj\'{\i}a, h y h$1$ alturas de entrepiso. 

%%%%%%%%%%%%%%%%%%%%%%%%%%%%%%%%%%%%%%%%%%%%%%%%%%%%%%%%%%%%%%%%%%%%%%%%%%%%%%%%%%%%%%%%%%%%%%%%%%%%%%%%%%%%%%%%%%%%%%%%%%%%%%%%%%%%%%%%%%%%%%%%%%%%%%%%%%%%